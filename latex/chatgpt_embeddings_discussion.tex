% Options for packages loaded elsewhere
\PassOptionsToPackage{unicode}{hyperref}
\PassOptionsToPackage{hyphens}{url}
\PassOptionsToPackage{dvipsnames,svgnames,x11names}{xcolor}
%
\documentclass[
  letterpaper,
  DIV=11,
  numbers=noendperiod]{scrartcl}

\usepackage{amsmath,amssymb}
\usepackage{iftex}
\ifPDFTeX
  \usepackage[T1]{fontenc}
  \usepackage[utf8]{inputenc}
  \usepackage{textcomp} % provide euro and other symbols
\else % if luatex or xetex
  \usepackage{unicode-math}
  \defaultfontfeatures{Scale=MatchLowercase}
  \defaultfontfeatures[\rmfamily]{Ligatures=TeX,Scale=1}
\fi
\usepackage{lmodern}
\ifPDFTeX\else  
    % xetex/luatex font selection
\fi
% Use upquote if available, for straight quotes in verbatim environments
\IfFileExists{upquote.sty}{\usepackage{upquote}}{}
\IfFileExists{microtype.sty}{% use microtype if available
  \usepackage[]{microtype}
  \UseMicrotypeSet[protrusion]{basicmath} % disable protrusion for tt fonts
}{}
\makeatletter
\@ifundefined{KOMAClassName}{% if non-KOMA class
  \IfFileExists{parskip.sty}{%
    \usepackage{parskip}
  }{% else
    \setlength{\parindent}{0pt}
    \setlength{\parskip}{6pt plus 2pt minus 1pt}}
}{% if KOMA class
  \KOMAoptions{parskip=half}}
\makeatother
\usepackage{xcolor}
\setlength{\emergencystretch}{3em} % prevent overfull lines
\setcounter{secnumdepth}{-\maxdimen} % remove section numbering
% Make \paragraph and \subparagraph free-standing
\makeatletter
\ifx\paragraph\undefined\else
  \let\oldparagraph\paragraph
  \renewcommand{\paragraph}{
    \@ifstar
      \xxxParagraphStar
      \xxxParagraphNoStar
  }
  \newcommand{\xxxParagraphStar}[1]{\oldparagraph*{#1}\mbox{}}
  \newcommand{\xxxParagraphNoStar}[1]{\oldparagraph{#1}\mbox{}}
\fi
\ifx\subparagraph\undefined\else
  \let\oldsubparagraph\subparagraph
  \renewcommand{\subparagraph}{
    \@ifstar
      \xxxSubParagraphStar
      \xxxSubParagraphNoStar
  }
  \newcommand{\xxxSubParagraphStar}[1]{\oldsubparagraph*{#1}\mbox{}}
  \newcommand{\xxxSubParagraphNoStar}[1]{\oldsubparagraph{#1}\mbox{}}
\fi
\makeatother

\usepackage{color}
\usepackage{fancyvrb}
\newcommand{\VerbBar}{|}
\newcommand{\VERB}{\Verb[commandchars=\\\{\}]}
\DefineVerbatimEnvironment{Highlighting}{Verbatim}{commandchars=\\\{\}}
% Add ',fontsize=\small' for more characters per line
\usepackage{framed}
\definecolor{shadecolor}{RGB}{241,243,245}
\newenvironment{Shaded}{\begin{snugshade}}{\end{snugshade}}
\newcommand{\AlertTok}[1]{\textcolor[rgb]{0.68,0.00,0.00}{#1}}
\newcommand{\AnnotationTok}[1]{\textcolor[rgb]{0.37,0.37,0.37}{#1}}
\newcommand{\AttributeTok}[1]{\textcolor[rgb]{0.40,0.45,0.13}{#1}}
\newcommand{\BaseNTok}[1]{\textcolor[rgb]{0.68,0.00,0.00}{#1}}
\newcommand{\BuiltInTok}[1]{\textcolor[rgb]{0.00,0.23,0.31}{#1}}
\newcommand{\CharTok}[1]{\textcolor[rgb]{0.13,0.47,0.30}{#1}}
\newcommand{\CommentTok}[1]{\textcolor[rgb]{0.37,0.37,0.37}{#1}}
\newcommand{\CommentVarTok}[1]{\textcolor[rgb]{0.37,0.37,0.37}{\textit{#1}}}
\newcommand{\ConstantTok}[1]{\textcolor[rgb]{0.56,0.35,0.01}{#1}}
\newcommand{\ControlFlowTok}[1]{\textcolor[rgb]{0.00,0.23,0.31}{\textbf{#1}}}
\newcommand{\DataTypeTok}[1]{\textcolor[rgb]{0.68,0.00,0.00}{#1}}
\newcommand{\DecValTok}[1]{\textcolor[rgb]{0.68,0.00,0.00}{#1}}
\newcommand{\DocumentationTok}[1]{\textcolor[rgb]{0.37,0.37,0.37}{\textit{#1}}}
\newcommand{\ErrorTok}[1]{\textcolor[rgb]{0.68,0.00,0.00}{#1}}
\newcommand{\ExtensionTok}[1]{\textcolor[rgb]{0.00,0.23,0.31}{#1}}
\newcommand{\FloatTok}[1]{\textcolor[rgb]{0.68,0.00,0.00}{#1}}
\newcommand{\FunctionTok}[1]{\textcolor[rgb]{0.28,0.35,0.67}{#1}}
\newcommand{\ImportTok}[1]{\textcolor[rgb]{0.00,0.46,0.62}{#1}}
\newcommand{\InformationTok}[1]{\textcolor[rgb]{0.37,0.37,0.37}{#1}}
\newcommand{\KeywordTok}[1]{\textcolor[rgb]{0.00,0.23,0.31}{\textbf{#1}}}
\newcommand{\NormalTok}[1]{\textcolor[rgb]{0.00,0.23,0.31}{#1}}
\newcommand{\OperatorTok}[1]{\textcolor[rgb]{0.37,0.37,0.37}{#1}}
\newcommand{\OtherTok}[1]{\textcolor[rgb]{0.00,0.23,0.31}{#1}}
\newcommand{\PreprocessorTok}[1]{\textcolor[rgb]{0.68,0.00,0.00}{#1}}
\newcommand{\RegionMarkerTok}[1]{\textcolor[rgb]{0.00,0.23,0.31}{#1}}
\newcommand{\SpecialCharTok}[1]{\textcolor[rgb]{0.37,0.37,0.37}{#1}}
\newcommand{\SpecialStringTok}[1]{\textcolor[rgb]{0.13,0.47,0.30}{#1}}
\newcommand{\StringTok}[1]{\textcolor[rgb]{0.13,0.47,0.30}{#1}}
\newcommand{\VariableTok}[1]{\textcolor[rgb]{0.07,0.07,0.07}{#1}}
\newcommand{\VerbatimStringTok}[1]{\textcolor[rgb]{0.13,0.47,0.30}{#1}}
\newcommand{\WarningTok}[1]{\textcolor[rgb]{0.37,0.37,0.37}{\textit{#1}}}

\providecommand{\tightlist}{%
  \setlength{\itemsep}{0pt}\setlength{\parskip}{0pt}}\usepackage{longtable,booktabs,array}
\usepackage{calc} % for calculating minipage widths
% Correct order of tables after \paragraph or \subparagraph
\usepackage{etoolbox}
\makeatletter
\patchcmd\longtable{\par}{\if@noskipsec\mbox{}\fi\par}{}{}
\makeatother
% Allow footnotes in longtable head/foot
\IfFileExists{footnotehyper.sty}{\usepackage{footnotehyper}}{\usepackage{footnote}}
\makesavenoteenv{longtable}
\usepackage{graphicx}
\makeatletter
\newsavebox\pandoc@box
\newcommand*\pandocbounded[1]{% scales image to fit in text height/width
  \sbox\pandoc@box{#1}%
  \Gscale@div\@tempa{\textheight}{\dimexpr\ht\pandoc@box+\dp\pandoc@box\relax}%
  \Gscale@div\@tempb{\linewidth}{\wd\pandoc@box}%
  \ifdim\@tempb\p@<\@tempa\p@\let\@tempa\@tempb\fi% select the smaller of both
  \ifdim\@tempa\p@<\p@\scalebox{\@tempa}{\usebox\pandoc@box}%
  \else\usebox{\pandoc@box}%
  \fi%
}
% Set default figure placement to htbp
\def\fps@figure{htbp}
\makeatother

\KOMAoption{captions}{tableheading}
\makeatletter
\@ifpackageloaded{caption}{}{\usepackage{caption}}
\AtBeginDocument{%
\ifdefined\contentsname
  \renewcommand*\contentsname{Table of contents}
\else
  \newcommand\contentsname{Table of contents}
\fi
\ifdefined\listfigurename
  \renewcommand*\listfigurename{List of Figures}
\else
  \newcommand\listfigurename{List of Figures}
\fi
\ifdefined\listtablename
  \renewcommand*\listtablename{List of Tables}
\else
  \newcommand\listtablename{List of Tables}
\fi
\ifdefined\figurename
  \renewcommand*\figurename{Figure}
\else
  \newcommand\figurename{Figure}
\fi
\ifdefined\tablename
  \renewcommand*\tablename{Table}
\else
  \newcommand\tablename{Table}
\fi
}
\@ifpackageloaded{float}{}{\usepackage{float}}
\floatstyle{ruled}
\@ifundefined{c@chapter}{\newfloat{codelisting}{h}{lop}}{\newfloat{codelisting}{h}{lop}[chapter]}
\floatname{codelisting}{Listing}
\newcommand*\listoflistings{\listof{codelisting}{List of Listings}}
\makeatother
\makeatletter
\makeatother
\makeatletter
\@ifpackageloaded{caption}{}{\usepackage{caption}}
\@ifpackageloaded{subcaption}{}{\usepackage{subcaption}}
\makeatother

\usepackage{bookmark}

\IfFileExists{xurl.sty}{\usepackage{xurl}}{} % add URL line breaks if available
\urlstyle{same} % disable monospaced font for URLs
\hypersetup{
  pdftitle={ChatGPT API and Word Embeddings},
  colorlinks=true,
  linkcolor={blue},
  filecolor={Maroon},
  citecolor={Blue},
  urlcolor={Blue},
  pdfcreator={LaTeX via pandoc}}


\title{ChatGPT API and Word Embeddings}
\author{}
\date{}

\begin{document}
\maketitle


\section{ChatGPT API and Word
Embeddings}\label{chatgpt-api-and-word-embeddings}

\subsection{Initial Request: Accessing ChatGPT API and Retrieving
Embeddings}\label{initial-request-accessing-chatgpt-api-and-retrieving-embeddings}

\textbf{Human}: Please write code that will send text to ChatGPT 3.5
Turbo and return the vector embedding. Use my ChatGPT API key.

\textbf{Assistant}: I've created code that demonstrates how to use
OpenAI's API to both get responses from ChatGPT 3.5 Turbo and retrieve
embeddings for text. The code is working successfully:

\begin{Shaded}
\begin{Highlighting}[]
\CommentTok{\# Import necessary libraries}
\ImportTok{import}\NormalTok{ os}
\ImportTok{import}\NormalTok{ openai}
\ImportTok{import}\NormalTok{ json}

\CommentTok{\# Set up the OpenAI client with API key from environment variables}
\ControlFlowTok{if} \StringTok{\textquotesingle{}OPEN\_AI\_KEY\textquotesingle{}} \KeywordTok{in}\NormalTok{ os.environ:}
\NormalTok{    client }\OperatorTok{=}\NormalTok{ openai.OpenAI(api\_key}\OperatorTok{=}\NormalTok{os.environ[}\StringTok{\textquotesingle{}OPEN\_AI\_KEY\textquotesingle{}}\NormalTok{])}
    \BuiltInTok{print}\NormalTok{(}\StringTok{"OpenAI client initialized with API key from environment variables."}\NormalTok{)}
\ControlFlowTok{else}\NormalTok{:}
    \BuiltInTok{print}\NormalTok{(}\StringTok{"OpenAI API key not found in environment variables. Please provide your API key."}\NormalTok{)}
    
\CommentTok{\# Function to get a response from ChatGPT 3.5 Turbo}
\KeywordTok{def}\NormalTok{ get\_chatgpt\_response(prompt):}
    \ControlFlowTok{try}\NormalTok{:}
\NormalTok{        response }\OperatorTok{=}\NormalTok{ client.chat.completions.create(}
\NormalTok{            model}\OperatorTok{=}\StringTok{"gpt{-}3.5{-}turbo"}\NormalTok{,}
\NormalTok{            messages}\OperatorTok{=}\NormalTok{[}
\NormalTok{                \{}\StringTok{"role"}\NormalTok{: }\StringTok{"system"}\NormalTok{, }\StringTok{"content"}\NormalTok{: }\StringTok{"You are a helpful assistant."}\NormalTok{\},}
\NormalTok{                \{}\StringTok{"role"}\NormalTok{: }\StringTok{"user"}\NormalTok{, }\StringTok{"content"}\NormalTok{: prompt\}}
\NormalTok{            ]}
\NormalTok{        )}
        \ControlFlowTok{return}\NormalTok{ response.choices[}\DecValTok{0}\NormalTok{].message.content}
    \ControlFlowTok{except} \PreprocessorTok{Exception} \ImportTok{as}\NormalTok{ e:}
        \ControlFlowTok{return} \SpecialStringTok{f"Error: }\SpecialCharTok{\{}\BuiltInTok{str}\NormalTok{(e)}\SpecialCharTok{\}}\SpecialStringTok{"}

\CommentTok{\# Function to get embeddings for a text}
\KeywordTok{def}\NormalTok{ get\_embedding(text):}
    \ControlFlowTok{try}\NormalTok{:}
\NormalTok{        response }\OperatorTok{=}\NormalTok{ client.embeddings.create(}
\NormalTok{            model}\OperatorTok{=}\StringTok{"text{-}embedding{-}ada{-}002"}\NormalTok{,}
            \BuiltInTok{input}\OperatorTok{=}\NormalTok{text}
\NormalTok{        )}
        \ControlFlowTok{return}\NormalTok{ response.data[}\DecValTok{0}\NormalTok{].embedding}
    \ControlFlowTok{except} \PreprocessorTok{Exception} \ImportTok{as}\NormalTok{ e:}
        \ControlFlowTok{return} \SpecialStringTok{f"Error: }\SpecialCharTok{\{}\BuiltInTok{str}\NormalTok{(e)}\SpecialCharTok{\}}\SpecialStringTok{"}

\CommentTok{\# Example usage}
\ControlFlowTok{if} \StringTok{\textquotesingle{}OPEN\_AI\_KEY\textquotesingle{}} \KeywordTok{in}\NormalTok{ os.environ:}
    \CommentTok{\# Example text}
\NormalTok{    example\_text }\OperatorTok{=} \StringTok{"Hello, world! This is a test for embeddings."}
    
    \CommentTok{\# Get response from ChatGPT}
    \BuiltInTok{print}\NormalTok{(}\StringTok{"Getting response from ChatGPT 3.5 Turbo..."}\NormalTok{)}
\NormalTok{    response }\OperatorTok{=}\NormalTok{ get\_chatgpt\_response(example\_text)}
    \BuiltInTok{print}\NormalTok{(}\StringTok{"}\CharTok{\textbackslash{}n}\StringTok{ChatGPT Response:"}\NormalTok{)}
    \BuiltInTok{print}\NormalTok{(response)}
    
    \CommentTok{\# Get embedding}
    \BuiltInTok{print}\NormalTok{(}\StringTok{"}\CharTok{\textbackslash{}n}\StringTok{Getting embedding for the text..."}\NormalTok{)}
\NormalTok{    embedding }\OperatorTok{=}\NormalTok{ get\_embedding(example\_text)}
    
    \CommentTok{\# Print the first 10 values of the embedding vector (to avoid printing the entire vector)}
    \BuiltInTok{print}\NormalTok{(}\StringTok{"}\CharTok{\textbackslash{}n}\StringTok{Embedding (first 10 values):"}\NormalTok{)}
    \ControlFlowTok{if} \BuiltInTok{isinstance}\NormalTok{(embedding, }\BuiltInTok{list}\NormalTok{):}
        \BuiltInTok{print}\NormalTok{(embedding[:}\DecValTok{10}\NormalTok{])}
        \BuiltInTok{print}\NormalTok{(}\SpecialStringTok{f"Total embedding dimensions: }\SpecialCharTok{\{}\BuiltInTok{len}\NormalTok{(embedding)}\SpecialCharTok{\}}\SpecialStringTok{"}\NormalTok{)}
    \ControlFlowTok{else}\NormalTok{:}
        \BuiltInTok{print}\NormalTok{(embedding)  }\CommentTok{\# This will print the error message if there was an error}
\ControlFlowTok{else}\NormalTok{:}
    \BuiltInTok{print}\NormalTok{(}\StringTok{"Cannot demonstrate API calls without an API key."}\NormalTok{)}
\end{Highlighting}
\end{Shaded}

The code includes two main functions: 1.
\texttt{get\_chatgpt\_response()} - Sends text to ChatGPT and returns
its response 2. \texttt{get\_embedding()} - Gets the embedding vector
for a given text

When testing with the example text ``Hello, world! This is a test for
embeddings.'', the embedding is a 1536-dimensional vector that
represents the semantic meaning of the text in a way that can be used
for various NLP tasks like similarity comparison, clustering, or
classification.

\subsection{Word Embeddings for King, Woman, Man,
Queen}\label{word-embeddings-for-king-woman-man-queen}

\textbf{Human}: Please send the following words to chatgpt one at a
time, get the embedding for each, and assign each vector to a column of
a dataframe with the word as the column name: ``king'', ``woman'',
``man'', ``queen''.

\textbf{Assistant}: I've successfully retrieved the embeddings for each
of the words ``king'', ``woman'', ``man'', and ``queen'' and created a
dataframe with each word as a column name. Each column contains the
1536-dimensional embedding vector for that word.

\begin{Shaded}
\begin{Highlighting}[]
\CommentTok{\# Import necessary libraries}
\ImportTok{import}\NormalTok{ os}
\ImportTok{import}\NormalTok{ openai}
\ImportTok{import}\NormalTok{ pandas }\ImportTok{as}\NormalTok{ pd}
\ImportTok{import}\NormalTok{ numpy }\ImportTok{as}\NormalTok{ np}
\ImportTok{from}\NormalTok{ tqdm }\ImportTok{import}\NormalTok{ tqdm}

\CommentTok{\# Set up the OpenAI client with API key from environment variables}
\ControlFlowTok{if} \StringTok{\textquotesingle{}OPEN\_AI\_KEY\textquotesingle{}} \KeywordTok{in}\NormalTok{ os.environ:}
\NormalTok{    client }\OperatorTok{=}\NormalTok{ openai.OpenAI(api\_key}\OperatorTok{=}\NormalTok{os.environ[}\StringTok{\textquotesingle{}OPEN\_AI\_KEY\textquotesingle{}}\NormalTok{])}
    \BuiltInTok{print}\NormalTok{(}\StringTok{"OpenAI client initialized with API key from environment variables."}\NormalTok{)}
\ControlFlowTok{else}\NormalTok{:}
    \BuiltInTok{print}\NormalTok{(}\StringTok{"OpenAI API key not found in environment variables. Please provide your API key."}\NormalTok{)}

\CommentTok{\# Function to get embeddings for a text}
\KeywordTok{def}\NormalTok{ get\_embedding(text):}
    \ControlFlowTok{try}\NormalTok{:}
\NormalTok{        response }\OperatorTok{=}\NormalTok{ client.embeddings.create(}
\NormalTok{            model}\OperatorTok{=}\StringTok{"text{-}embedding{-}ada{-}002"}\NormalTok{,}
            \BuiltInTok{input}\OperatorTok{=}\NormalTok{text}
\NormalTok{        )}
        \ControlFlowTok{return}\NormalTok{ response.data[}\DecValTok{0}\NormalTok{].embedding}
    \ControlFlowTok{except} \PreprocessorTok{Exception} \ImportTok{as}\NormalTok{ e:}
        \ControlFlowTok{return} \SpecialStringTok{f"Error: }\SpecialCharTok{\{}\BuiltInTok{str}\NormalTok{(e)}\SpecialCharTok{\}}\SpecialStringTok{"}

\CommentTok{\# Words to get embeddings for}
\NormalTok{words }\OperatorTok{=}\NormalTok{ [}\StringTok{"king"}\NormalTok{, }\StringTok{"woman"}\NormalTok{, }\StringTok{"man"}\NormalTok{, }\StringTok{"queen"}\NormalTok{]}

\CommentTok{\# Create a dictionary to store embeddings}
\NormalTok{embeddings\_dict }\OperatorTok{=}\NormalTok{ \{\}}

\CommentTok{\# Get embeddings for each word}
\ControlFlowTok{for}\NormalTok{ word }\KeywordTok{in}\NormalTok{ tqdm(words, desc}\OperatorTok{=}\StringTok{"Getting embeddings"}\NormalTok{):}
\NormalTok{    embedding }\OperatorTok{=}\NormalTok{ get\_embedding(word)}
    \ControlFlowTok{if} \BuiltInTok{isinstance}\NormalTok{(embedding, }\BuiltInTok{list}\NormalTok{):}
\NormalTok{        embeddings\_dict[word] }\OperatorTok{=}\NormalTok{ embedding}
    \ControlFlowTok{else}\NormalTok{:}
        \BuiltInTok{print}\NormalTok{(}\SpecialStringTok{f"Error getting embedding for \textquotesingle{}}\SpecialCharTok{\{}\NormalTok{word}\SpecialCharTok{\}}\SpecialStringTok{\textquotesingle{}: }\SpecialCharTok{\{}\NormalTok{embedding}\SpecialCharTok{\}}\SpecialStringTok{"}\NormalTok{)}

\CommentTok{\# Check if we have embeddings for all words}
\ControlFlowTok{if} \BuiltInTok{len}\NormalTok{(embeddings\_dict) }\OperatorTok{==} \BuiltInTok{len}\NormalTok{(words):}
    \CommentTok{\# Create a DataFrame with embeddings}
\NormalTok{    df }\OperatorTok{=}\NormalTok{ pd.DataFrame(embeddings\_dict)}
    
    \CommentTok{\# Display information about the DataFrame}
    \BuiltInTok{print}\NormalTok{(}\SpecialStringTok{f"}\CharTok{\textbackslash{}n}\SpecialStringTok{Created DataFrame with shape: }\SpecialCharTok{\{}\NormalTok{df}\SpecialCharTok{.}\NormalTok{shape}\SpecialCharTok{\}}\SpecialStringTok{"}\NormalTok{)}
    \BuiltInTok{print}\NormalTok{(}\SpecialStringTok{f"Column names: }\SpecialCharTok{\{}\NormalTok{df}\SpecialCharTok{.}\NormalTok{columns}\SpecialCharTok{.}\NormalTok{tolist()}\SpecialCharTok{\}}\SpecialStringTok{"}\NormalTok{)}
    
    \CommentTok{\# Display the first few rows of the DataFrame}
    \BuiltInTok{print}\NormalTok{(}\StringTok{"}\CharTok{\textbackslash{}n}\StringTok{First 5 rows of the DataFrame:"}\NormalTok{)}
    \BuiltInTok{print}\NormalTok{(df.head())}
\ControlFlowTok{else}\NormalTok{:}
    \BuiltInTok{print}\NormalTok{(}\StringTok{"Could not get embeddings for all words."}\NormalTok{)}
\end{Highlighting}
\end{Shaded}

I also created a visualization of these word embeddings after reducing
them to 2 dimensions using PCA:

\begin{Shaded}
\begin{Highlighting}[]
\CommentTok{\# Let\textquotesingle{}s visualize the relationships between these word embeddings}
\ImportTok{import}\NormalTok{ matplotlib.pyplot }\ImportTok{as}\NormalTok{ plt}
\ImportTok{from}\NormalTok{ sklearn.decomposition }\ImportTok{import}\NormalTok{ PCA}

\CommentTok{\# Convert the DataFrame to a format suitable for PCA}
\CommentTok{\# Each row is a word, each column is a dimension of the embedding}
\NormalTok{embeddings\_matrix }\OperatorTok{=}\NormalTok{ df.T.values}

\CommentTok{\# Apply PCA to reduce to 2 dimensions for visualization}
\NormalTok{pca }\OperatorTok{=}\NormalTok{ PCA(n\_components}\OperatorTok{=}\DecValTok{2}\NormalTok{)}
\NormalTok{reduced\_embeddings }\OperatorTok{=}\NormalTok{ pca.fit\_transform(embeddings\_matrix)}

\CommentTok{\# Create a scatter plot}
\NormalTok{plt.figure(figsize}\OperatorTok{=}\NormalTok{(}\DecValTok{10}\NormalTok{, }\DecValTok{8}\NormalTok{))}
\NormalTok{plt.scatter(reduced\_embeddings[:, }\DecValTok{0}\NormalTok{], reduced\_embeddings[:, }\DecValTok{1}\NormalTok{], c}\OperatorTok{=}\StringTok{\textquotesingle{}blue\textquotesingle{}}\NormalTok{, alpha}\OperatorTok{=}\FloatTok{0.7}\NormalTok{)}

\CommentTok{\# Add labels for each point}
\ControlFlowTok{for}\NormalTok{ i, word }\KeywordTok{in} \BuiltInTok{enumerate}\NormalTok{(words):}
\NormalTok{    plt.annotate(word, (reduced\_embeddings[i, }\DecValTok{0}\NormalTok{], reduced\_embeddings[i, }\DecValTok{1}\NormalTok{]), }
\NormalTok{                 fontsize}\OperatorTok{=}\DecValTok{12}\NormalTok{, ha}\OperatorTok{=}\StringTok{\textquotesingle{}center\textquotesingle{}}\NormalTok{, va}\OperatorTok{=}\StringTok{\textquotesingle{}bottom\textquotesingle{}}\NormalTok{)}

\CommentTok{\# Add arrows to show relationships}
\CommentTok{\# For example, the famous king {-} man + woman = queen relationship}
\NormalTok{plt.arrow(reduced\_embeddings[}\DecValTok{0}\NormalTok{, }\DecValTok{0}\NormalTok{], reduced\_embeddings[}\DecValTok{0}\NormalTok{, }\DecValTok{1}\NormalTok{],  }\CommentTok{\# king}
\NormalTok{          reduced\_embeddings[}\DecValTok{2}\NormalTok{, }\DecValTok{0}\NormalTok{] }\OperatorTok{{-}}\NormalTok{ reduced\_embeddings[}\DecValTok{0}\NormalTok{, }\DecValTok{0}\NormalTok{],  }\CommentTok{\# to man}
\NormalTok{          reduced\_embeddings[}\DecValTok{2}\NormalTok{, }\DecValTok{1}\NormalTok{] }\OperatorTok{{-}}\NormalTok{ reduced\_embeddings[}\DecValTok{0}\NormalTok{, }\DecValTok{1}\NormalTok{],}
\NormalTok{          color}\OperatorTok{=}\StringTok{\textquotesingle{}red\textquotesingle{}}\NormalTok{, alpha}\OperatorTok{=}\FloatTok{0.4}\NormalTok{, width}\OperatorTok{=}\FloatTok{0.01}\NormalTok{, head\_width}\OperatorTok{=}\FloatTok{0.05}\NormalTok{)}

\NormalTok{plt.arrow(reduced\_embeddings[}\DecValTok{1}\NormalTok{, }\DecValTok{0}\NormalTok{], reduced\_embeddings[}\DecValTok{1}\NormalTok{, }\DecValTok{1}\NormalTok{],  }\CommentTok{\# woman}
\NormalTok{          reduced\_embeddings[}\DecValTok{3}\NormalTok{, }\DecValTok{0}\NormalTok{] }\OperatorTok{{-}}\NormalTok{ reduced\_embeddings[}\DecValTok{1}\NormalTok{, }\DecValTok{0}\NormalTok{],  }\CommentTok{\# to queen}
\NormalTok{          reduced\_embeddings[}\DecValTok{3}\NormalTok{, }\DecValTok{1}\NormalTok{] }\OperatorTok{{-}}\NormalTok{ reduced\_embeddings[}\DecValTok{1}\NormalTok{, }\DecValTok{1}\NormalTok{],}
\NormalTok{          color}\OperatorTok{=}\StringTok{\textquotesingle{}green\textquotesingle{}}\NormalTok{, alpha}\OperatorTok{=}\FloatTok{0.4}\NormalTok{, width}\OperatorTok{=}\FloatTok{0.01}\NormalTok{, head\_width}\OperatorTok{=}\FloatTok{0.05}\NormalTok{)}

\NormalTok{plt.title(}\StringTok{\textquotesingle{}Word Embeddings Reduced to 2D using PCA\textquotesingle{}}\NormalTok{, fontsize}\OperatorTok{=}\DecValTok{14}\NormalTok{)}
\NormalTok{plt.grid(}\VariableTok{True}\NormalTok{, alpha}\OperatorTok{=}\FloatTok{0.3}\NormalTok{)}
\NormalTok{plt.tight\_layout()}
\NormalTok{plt.show()}

\CommentTok{\# Let\textquotesingle{}s also calculate some vector arithmetic to test the famous relationship:}
\CommentTok{\# king {-} man + woman ≈ queen}
\NormalTok{king\_vec }\OperatorTok{=}\NormalTok{ df[}\StringTok{\textquotesingle{}king\textquotesingle{}}\NormalTok{].values}
\NormalTok{man\_vec }\OperatorTok{=}\NormalTok{ df[}\StringTok{\textquotesingle{}man\textquotesingle{}}\NormalTok{].values}
\NormalTok{woman\_vec }\OperatorTok{=}\NormalTok{ df[}\StringTok{\textquotesingle{}woman\textquotesingle{}}\NormalTok{].values}
\NormalTok{queen\_vec }\OperatorTok{=}\NormalTok{ df[}\StringTok{\textquotesingle{}queen\textquotesingle{}}\NormalTok{].values}

\CommentTok{\# Calculate king {-} man + woman}
\NormalTok{result\_vec }\OperatorTok{=}\NormalTok{ king\_vec }\OperatorTok{{-}}\NormalTok{ man\_vec }\OperatorTok{+}\NormalTok{ woman\_vec}

\CommentTok{\# Calculate cosine similarity between result and queen}
\ImportTok{from}\NormalTok{ sklearn.metrics.pairwise }\ImportTok{import}\NormalTok{ cosine\_similarity}
\ImportTok{import}\NormalTok{ numpy }\ImportTok{as}\NormalTok{ np}

\CommentTok{\# Reshape vectors for cosine\_similarity function}
\NormalTok{result\_vec\_reshaped }\OperatorTok{=}\NormalTok{ result\_vec.reshape(}\DecValTok{1}\NormalTok{, }\OperatorTok{{-}}\DecValTok{1}\NormalTok{)}
\NormalTok{queen\_vec\_reshaped }\OperatorTok{=}\NormalTok{ queen\_vec.reshape(}\DecValTok{1}\NormalTok{, }\OperatorTok{{-}}\DecValTok{1}\NormalTok{)}

\NormalTok{similarity }\OperatorTok{=}\NormalTok{ cosine\_similarity(result\_vec\_reshaped, queen\_vec\_reshaped)[}\DecValTok{0}\NormalTok{][}\DecValTok{0}\NormalTok{]}
\BuiltInTok{print}\NormalTok{(}\SpecialStringTok{f"Cosine similarity between (king {-} man + woman) and queen: }\SpecialCharTok{\{}\NormalTok{similarity}\SpecialCharTok{:.4f\}}\SpecialStringTok{"}\NormalTok{)}

\CommentTok{\# Let\textquotesingle{}s also calculate other similarities for comparison}
\NormalTok{king\_man\_similarity }\OperatorTok{=}\NormalTok{ cosine\_similarity(king\_vec.reshape(}\DecValTok{1}\NormalTok{, }\OperatorTok{{-}}\DecValTok{1}\NormalTok{), man\_vec.reshape(}\DecValTok{1}\NormalTok{, }\OperatorTok{{-}}\DecValTok{1}\NormalTok{))[}\DecValTok{0}\NormalTok{][}\DecValTok{0}\NormalTok{]}
\NormalTok{woman\_queen\_similarity }\OperatorTok{=}\NormalTok{ cosine\_similarity(woman\_vec.reshape(}\DecValTok{1}\NormalTok{, }\OperatorTok{{-}}\DecValTok{1}\NormalTok{), queen\_vec.reshape(}\DecValTok{1}\NormalTok{, }\OperatorTok{{-}}\DecValTok{1}\NormalTok{))[}\DecValTok{0}\NormalTok{][}\DecValTok{0}\NormalTok{]}
\NormalTok{king\_queen\_similarity }\OperatorTok{=}\NormalTok{ cosine\_similarity(king\_vec.reshape(}\DecValTok{1}\NormalTok{, }\OperatorTok{{-}}\DecValTok{1}\NormalTok{), queen\_vec.reshape(}\DecValTok{1}\NormalTok{, }\OperatorTok{{-}}\DecValTok{1}\NormalTok{))[}\DecValTok{0}\NormalTok{][}\DecValTok{0}\NormalTok{]}
\NormalTok{man\_woman\_similarity }\OperatorTok{=}\NormalTok{ cosine\_similarity(man\_vec.reshape(}\DecValTok{1}\NormalTok{, }\OperatorTok{{-}}\DecValTok{1}\NormalTok{), woman\_vec.reshape(}\DecValTok{1}\NormalTok{, }\OperatorTok{{-}}\DecValTok{1}\NormalTok{))[}\DecValTok{0}\NormalTok{][}\DecValTok{0}\NormalTok{]}

\BuiltInTok{print}\NormalTok{(}\SpecialStringTok{f"Similarity between king and man: }\SpecialCharTok{\{}\NormalTok{king\_man\_similarity}\SpecialCharTok{:.4f\}}\SpecialStringTok{"}\NormalTok{)}
\BuiltInTok{print}\NormalTok{(}\SpecialStringTok{f"Similarity between woman and queen: }\SpecialCharTok{\{}\NormalTok{woman\_queen\_similarity}\SpecialCharTok{:.4f\}}\SpecialStringTok{"}\NormalTok{)}
\BuiltInTok{print}\NormalTok{(}\SpecialStringTok{f"Similarity between king and queen: }\SpecialCharTok{\{}\NormalTok{king\_queen\_similarity}\SpecialCharTok{:.4f\}}\SpecialStringTok{"}\NormalTok{)}
\BuiltInTok{print}\NormalTok{(}\SpecialStringTok{f"Similarity between man and woman: }\SpecialCharTok{\{}\NormalTok{man\_woman\_similarity}\SpecialCharTok{:.4f\}}\SpecialStringTok{"}\NormalTok{)}
\end{Highlighting}
\end{Shaded}

\begin{figure}[H]

{\centering \pandocbounded{\includegraphics[keepaspectratio]{word_embeddings_pca.png}}

}

\caption{Word Embeddings PCA}

\end{figure}%

To test the famous word analogy relationship (king - man + woman ≈
queen), I calculated the cosine similarity between the result of this
vector arithmetic and the actual queen vector. The high similarity score
(0.8844) confirms that the embeddings capture the semantic relationship
between these words quite well.

These results demonstrate how word embeddings capture meaningful
semantic relationships between words. The high similarity between (king
- man + woman) and queen shows that the embedding space has encoded
gender relationships in a consistent way, which is one of the classic
examples of how word embeddings work.

\subsection{Saving the DataFrame to
Excel}\label{saving-the-dataframe-to-excel}

\textbf{Human}: save the dataframe to an excel file and provide a link.

\textbf{Assistant}: The word embeddings dataframe has been successfully
saved to an Excel file. You can download the file from the link below.
The file contains all 1536 dimensions of the embeddings for each of the
four words (king, woman, man, and queen), with each word as a column and
each row representing one dimension of the embedding vector.

\begin{Shaded}
\begin{Highlighting}[]
\CommentTok{\# Save the DataFrame to an Excel file}
\ImportTok{import}\NormalTok{ pandas }\ImportTok{as}\NormalTok{ pd}

\CommentTok{\# Check if the DataFrame exists}
\ControlFlowTok{try}\NormalTok{:}
    \CommentTok{\# Save the DataFrame to an Excel file}
\NormalTok{    excel\_filename }\OperatorTok{=} \StringTok{"word\_embeddings.xlsx"}
\NormalTok{    df.to\_excel(excel\_filename)}
    \BuiltInTok{print}\NormalTok{(}\SpecialStringTok{f"DataFrame saved to }\SpecialCharTok{\{}\NormalTok{excel\_filename}\SpecialCharTok{\}}\SpecialStringTok{"}\NormalTok{)}
    
    \CommentTok{\# Provide information about the file}
    \ImportTok{import}\NormalTok{ os}
\NormalTok{    file\_size }\OperatorTok{=}\NormalTok{ os.path.getsize(excel\_filename) }\OperatorTok{/} \DecValTok{1024}  \CommentTok{\# Size in KB}
    \BuiltInTok{print}\NormalTok{(}\SpecialStringTok{f"File size: }\SpecialCharTok{\{}\NormalTok{file\_size}\SpecialCharTok{:.2f\}}\SpecialStringTok{ KB"}\NormalTok{)}
    \BuiltInTok{print}\NormalTok{(}\SpecialStringTok{f"Download link: https://julius.ai/files?filename=}\SpecialCharTok{\{}\NormalTok{excel\_filename}\SpecialCharTok{\}}\SpecialStringTok{"}\NormalTok{)}
\ControlFlowTok{except} \PreprocessorTok{NameError}\NormalTok{:}
    \BuiltInTok{print}\NormalTok{(}\StringTok{"DataFrame \textquotesingle{}df\textquotesingle{} not found. Please make sure the DataFrame has been created."}\NormalTok{)}
\ControlFlowTok{except} \PreprocessorTok{Exception} \ImportTok{as}\NormalTok{ e:}
    \BuiltInTok{print}\NormalTok{(}\SpecialStringTok{f"Error saving DataFrame to Excel: }\SpecialCharTok{\{}\BuiltInTok{str}\NormalTok{(e)}\SpecialCharTok{\}}\SpecialStringTok{"}\NormalTok{)}
\end{Highlighting}
\end{Shaded}

Download link: https://julius.ai/files?filename=word\_embeddings.xlsx

\subsection{Creating a Quarto Markdown
File}\label{creating-a-quarto-markdown-file}

\textbf{Human}: Copy all of the discussion in this chat, including code
and the image, into a quarto markdown file and provide a link.

\textbf{Assistant}: I've created a Quarto markdown file that contains
the entire chat discussion, including all the code and the
visualization. The file is available for download at the link below.

The Quarto markdown file includes: - The initial code for accessing
ChatGPT API and retrieving embeddings - The code for creating word
embeddings for ``king'', ``woman'', ``man'', and ``queen'' - The
visualization of the word embeddings using PCA - The vector arithmetic
demonstrating the relationship between these words - The code for saving
the DataFrame to an Excel file

Download link:
https://julius.ai/files?filename=chatgpt\_embeddings\_discussion.qmd




\end{document}
