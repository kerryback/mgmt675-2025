\documentclass[10pt]{beamer}

\usepackage{hyperref}
\hypersetup{
    colorlinks=true,
    linkcolor=black,
    filecolor=black,
    urlcolor=blue,
    citecolor=black
}

\usetheme[progressbar=foot]{metropolis}
\usepackage{appendixnumberbeamer}

\usepackage{booktabs}
\usepackage[scale=2]{ccicons}

\usepackage{pgfplots}
\usepgfplotslibrary{dateplot}

\usepackage{xspace}
\usepackage{xcolor}

\DeclareMathOperator{\stdev}{stdev}
\DeclareMathOperator{\var}{var}
\DeclareMathOperator{\cov}{cov}
\DeclareMathOperator{\corr}{corr}
\DeclareMathOperator{\prob}{prob}
\DeclareMathOperator{\n}{n}
\DeclareMathOperator{\N}{N}
\DeclareMathOperator{\Cov}{Cov}

\newcommand{\D}{\mathrm{d}}
\newcommand{\E}{\mathrm{e}}
\newcommand{\mye}{\ensuremath{\mathsf{E}}}
\newcommand{\myreal}{\ensuremath{\mathbb{R}}}

\setbeamertemplate{frame footer}{MGMT 675}


\setbeamertemplate{title page}{
  \begin{centering}
    \begin{beamercolorbox}[sep=8pt,center]{title}
      \usebeamerfont{title}\inserttitle\par%
      \ifx\insertsubtitle\@empty%
      \else%
        \vskip0.25em%
        {\usebeamerfont{subtitle}\usebeamercolor[fg]{subtitle}\insertsubtitle\par}%
      \fi%     
    \end{beamercolorbox}%
    \vfill
    \begin{beamercolorbox}[sep=8pt,center]{date}
      \usebeamerfont{date}\insertdate
    \end{beamercolorbox}
    \vskip0.5em
    {\usebeamercolor[fg]{titlegraphic}\inserttitlegraphic\par}
  \end{centering}
}

\title{Cost of Equity Capital}
\subtitle{MGMT 675: AI-Assisted Financial Analysis}
\titlegraphic{\includegraphics[height=1cm]{../docs/RiceBusiness-transparent-logo-sm.png}}
\date{}
\begin{document}

\begin{frame}[plain]
\titlepage
\end{frame}

\begin{frame}{Outline}
\begin{itemize}
\item Overview of cost of equity capital
\item Data sources
  \begin{itemize}
  \item FRED
  \item Ken French's data library 
  \item Yahoo Finance
  \end{itemize}
\item Regressions
\end{itemize}
\end{frame}

\section{CAPM}
\begin{frame}{Cost of Equity Capital}
\begin{itemize}
\item According to the CAPM, the cost of equity capital is
$$\text{risk-free rate} + \text{beta} \times \text{market risk premium}$$
\item What to use for the risk-free rate?
\item How to estimate the market risk premium?
\item How to estimate beta?
\end{itemize}
\end{frame}

\section{Data Sources}
\begin{frame}{FRED}
\begin{itemize}
\item FRED = Federal Reserve Economic Data
\item \href{https://fred.stlouisfed.org/}{https://fred.stlouisfed.org/}
\item Easy way to get data is with pandas datareader
\item Ask Julius to get the current 10-year U.S. Treasury yield from FRED
\item Ask Julius to get the current 3-month U.S. Treasury yield from FRED
\end{itemize}
\end{frame}

\begin{frame}{Ken French's data library}
\begin{itemize}
\item \href{https://mba.tuck.dartmouth.edu/pages/faculty/ken.french/data_library.html}{French's website} at Dartmouth (actually DFA)
\item Easy way to get data is with pandas datareader
\item Ask Julius to list the data sets available
\item Ask Julius to get monthly Mkt-RF and RF beginning in 1926
\end{itemize}
\end{frame}

\begin{frame}{Yahoo Finance}
\begin{itemize}
\item \href{https://finance.yahoo.com/}{https://finance.yahoo.com/}
\item Yahoo computes split-adjusted prices (like everyone does) and also split and dividend adjusted prices (called adjusted close)
\item Percent change in adjusted close is capital gain + dividend yield
\item Easy way to get data is with yfinance
\end{itemize}
\end{frame}

\begin{frame}{yfinance version}
\begin{itemize}
\item yfinance recently started auto adjusting - when you ask for closing prices, you get the adjusted close by default
\item Ask Julius to use yfinance 0.2.54 (most recent version)
\item Ask Julius for closing prices (you'll get the adjusted close)
\item Compute percent changes to get returns
\item Example: Ask Julius to use yfinance 0.2.54 to get monthly closing prices for AAPL and to compute returns as percent changes
\end{itemize}
\end{frame}

\begin{frame}{Merging data}
\begin{itemize} 
\item Date formats
\begin{itemize}
\item Ask Julius what the date format is for the Mkt-RF and RF data
\item Ask Julius what the date format is for the AAPL returns
\item Ask Julius how to merge the two data sets
\item Tell Julius to merge them as it recommends
\end{itemize}
\item Decimal or percentage?  Ask Julius to convert the AAPL return to percentage or to convert Mkt-RF and RF to decimal.
\end{itemize}
\end{frame}

\section{Regression}
\begin{frame}{CAPM regression}
  \begin{itemize}
  \item We want to regress the excess stock return on the excess market return
  \item Conventional to use most recent 60 months of data
  \item Ask Julius to compute the AAPL return minus RF 
  \item Ask Julius to regress the excess AAPL return on Mkt-RF using the most recent 60 months for which both are available
  \item Ask Julius to show a scatter plot with the regression line
\end{itemize}
\end{frame}

\begin{frame}{Julius workflow}
    \begin{itemize}
    \item Create a Julius workflow in which the user inputs a ticker and specifies whether she wants to use the 3-month Treasury or 10-year Treasury as the risk-free rate.  
    \item The workflow should return the cost of equity capital and a scatter plot of the regression.
    \end{itemize}
\end{frame}
\end{document}