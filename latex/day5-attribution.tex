\documentclass[10pt]{beamer}

\usepackage{hyperref}
\hypersetup{
    colorlinks=true,
    linkcolor=black,
    filecolor=black,
    urlcolor=blue,
    citecolor=black
}

\usetheme[progressbar=foot]{metropolis}
\usepackage{appendixnumberbeamer}

\usepackage{booktabs}
\usepackage[scale=2]{ccicons}

\usepackage{pgfplots}
\usepgfplotslibrary{dateplot}

\usepackage{xspace}
\usepackage{xcolor}

\DeclareMathOperator{\stdev}{stdev}
\DeclareMathOperator{\var}{var}
\DeclareMathOperator{\cov}{cov}
\DeclareMathOperator{\corr}{corr}
\DeclareMathOperator{\prob}{prob}
\DeclareMathOperator{\n}{n}
\DeclareMathOperator{\N}{N}
\DeclareMathOperator{\Cov}{Cov}

\newcommand{\D}{\mathrm{d}}
\newcommand{\E}{\mathrm{e}}
\newcommand{\mye}{\ensuremath{\mathsf{E}}}
\newcommand{\myreal}{\ensuremath{\mathbb{R}}}

\setbeamertemplate{frame footer}{MGMT 675}


\setbeamertemplate{title page}{
  \begin{centering}
    \begin{beamercolorbox}[sep=8pt,center]{title}
      \usebeamerfont{title}\inserttitle\par%
      \ifx\insertsubtitle\@empty%
      \else%
        \vskip0.25em%
        {\usebeamerfont{subtitle}\usebeamercolor[fg]{subtitle}\insertsubtitle\par}%
      \fi%     
    \end{beamercolorbox}%
    \vfill
    \begin{beamercolorbox}[sep=8pt,center]{date}
      \usebeamerfont{date}\insertdate
    \end{beamercolorbox}
    \vskip0.5em
    {\usebeamercolor[fg]{titlegraphic}\inserttitlegraphic\par}
  \end{centering}
}

\title{Attribution of Fund Returns}
\subtitle{MGMT 675: AI-Assisted Financial Analysis}
\titlegraphic{\includegraphics[height=1cm]{../docs/RiceBusiness-transparent-logo-sm.png}}
\date{}
\begin{document}

\begin{frame}[plain]
\titlepage
\end{frame}

\begin{frame}{Outline}
\begin{itemize}
\item Mutual fund styles
\item Fama-French factors
\item Other factors
\item Julius workflow
\end{itemize}
\end{frame}

\section{Styles and Factors}

\begin{frame}{How to Evaluate a Mutual Fund?}
\begin{itemize}
\item Relative to the market?  Did it beat the market?
\item \pause Take into account the market exposure of the fund?  
\begin{itemize}
\item Example: a high beta fund should beat the market on average
\end{itemize}
\item Take into account the style of the fund?
\begin{itemize}
\item Is it reasonable to compare a small cap fund to a large cap fund?
\item Is it reasonable to compare a value fund to a growth fund?
\end{itemize}
\item \pause How to determine the fund's style?
\begin{itemize}
\item Holdings are reported quarterly, are just a snapshot at a point in time
\item Can we use returns to infer styles?
\end{itemize}
\end{itemize}
\end{frame}


\begin{frame}{Fama-French Regression and Alpha}
    \begin{align*}
        r_i - r_f = \alpha_i + \beta_{\text{Mkt-RF}}\text{Mkt-RF} + \beta_{\text{SMB}}\text{SMB} 
        + \beta_{\text{HML}}\text{HML} \\
        + \beta_{\text{CMA}}\text{CMA} + \beta_{\text{RMW}}\text{RMW} + \varepsilon_i 
        \end{align*}
\begin{itemize}
\item Coefficients are style exposures
\item Seeking alpha:
\begin{itemize}
\item Fama-French pricing model is:
$$\overline{r_i} - r_f = \beta_{\text{Mkt-RF}}\overline{\text{Mkt-RF}} + \beta_{\text{SMB}}\overline{\text{SMB}} + \beta_{\text{HML}}\overline{\text{HML}} + \beta_{\text{CMA}}\overline{\text{CMA}} + \beta_{\text{RMW}}\overline{\text{RMW}} $$
\item Fama-French pricing model is equivalent to: $\alpha_i=0$
\item $\alpha_i>0$ means outperformance, given style exposures.
\end{itemize}
\end{itemize}
\end{frame}

\begin{frame}{Other style factors}
\begin{itemize}
\item Past returns:
    \begin{itemize}
    \item Momentum: return over past 12 months excluding most recent month (high minus low)
\item Long-term reversal: return over past 5 years excluding most recent year (low minus high)
\item Short-term reversal: return over previous month (low minus high)
    \end{itemize}
\item Volatility and idiosyncratic volatility
\item Liquidity (volume, spreads)
\item Betting against beta: return of levered portfolio of low beta stocks minus de-levered portfolio of high beta stocks
\item Quality (profitability, growth, safety, payouts)
\end{itemize}
\end{frame}

\begin{frame}{Data sources}
    \begin{itemize}
    \item \href{https://mba.tuck.dartmouth.edu/pages/faculty/ken.french/data_library.html}{French's data library}
    \item \href{https://www.aqr.com/library/data-sets}{AQR's data library}
    \end{itemize}
\end{frame}

\section{Julius Workflow}

\begin{frame}{User prompts}
    \begin{itemize}
\item Ask user to provide the ticker of a mutual fund 
\item Provide a list of available factors and ask the user to select from the list (more than one typically)
    \end{itemize}
\end{frame}

\begin{frame}{Julius prompts}
\begin{itemize}
\item Get monthly closing prices for the mutual fund using yfinance 0.2.54 and compute returns as percent changes.
\item Get RF from French's data library.  Convert to decimal.
\item Get required factor data from French or AQR.  Convert French data to decimals.
\item Convert dates to compatible formats and merge the data.
\item Compute the excess mutual fund return by subtracting RF.
\item Regress the excess mutual fund return on the factors.  Report a summary of the regression.
\end{itemize}
\end{frame}

\begin{frame}{Plotting the results}
\begin{itemize}
\item Given the estimated betas, compute the factor contribution to the return each month as beta $\times$ factor return.
\item Compute the active return as the difference between the mutual fund excess return and the sum of the factor contributions.
\item Compound each factor contribution over time and compound the active return over time.
\item Plot the compounded active return and each of the compounded factor contributions in a single figure.
\end{itemize}
\end{frame}
\end{document}