\documentclass[10pt]{beamer}

\usepackage{hyperref}
\hypersetup{
    colorlinks=true,
    linkcolor=black,
    filecolor=black,
    urlcolor=blue,
    citecolor=black
}

\usetheme[progressbar=foot]{metropolis}
\usepackage{appendixnumberbeamer}

\usepackage{booktabs}
\usepackage[scale=2]{ccicons}

\usepackage{pgfplots}
\usepgfplotslibrary{dateplot}

\usepackage{xspace}
\usepackage{xcolor}

\DeclareMathOperator{\stdev}{stdev}
\DeclareMathOperator{\var}{var}
\DeclareMathOperator{\cov}{cov}
\DeclareMathOperator{\corr}{corr}
\DeclareMathOperator{\prob}{prob}
\DeclareMathOperator{\n}{n}
\DeclareMathOperator{\N}{N}
\DeclareMathOperator{\Cov}{Cov}

\newcommand{\D}{\mathrm{d}}
\newcommand{\E}{\mathrm{e}}
\newcommand{\mye}{\ensuremath{\mathsf{E}}}
\newcommand{\myreal}{\ensuremath{\mathbb{R}}}

\setbeamertemplate{frame footer}{MGMT 675}


\setbeamertemplate{title page}{
  \begin{centering}
    \begin{beamercolorbox}[sep=8pt,center]{title}
      \usebeamerfont{title}\inserttitle\par%
      \ifx\insertsubtitle\@empty%
      \else%
        \vskip0.25em%
        {\usebeamerfont{subtitle}\usebeamercolor[fg]{subtitle}\insertsubtitle\par}%
      \fi%     
    \end{beamercolorbox}%
    \vfill
    \begin{beamercolorbox}[sep=8pt,center]{date}
      \usebeamerfont{date}\insertdate
    \end{beamercolorbox}
    \vskip0.5em
    {\usebeamercolor[fg]{titlegraphic}\inserttitlegraphic\par}
  \end{centering}
}

\title{Financial Ratios}
\subtitle{MGMT 675: AI-Assisted Financial Analysis}
\titlegraphic{\includegraphics[height=1cm]{../docs/RiceBusiness-transparent-logo-sm.png}}
\date{}
\begin{document}

\begin{frame}[plain]
\titlepage
\end{frame}

\begin{frame}{Outline}
\begin{itemize}
\item Data sources
\item Key financial ratios
\item Julius workflow
\end{itemize}
\end{frame}

\section{Data Sources}

\begin{frame}{SEC EDGAR}
\begin{itemize}
\item All financial statement data ultimately comesfrom the SEC 
\item SEC EDGAR (Electronic Data Gathering, Analysis, and Retrieval) is the SEC's online database where companies submit  filings and the public can download filings. 
\item We can search it in a browser
\item We can do a bulk download
\item We can scrape it
\end{itemize}
\end{frame}

\begin{frame}{Yahoo Finance}
\begin{itemize}
\item The easiest way to get financial statement data is to use yfinance.
\item You'll get a summary of the statements.  Various items are lumped into standard categories.
\item The SEC provides more granular data.
\item Example: ask Julius to use yfinance 0.2.54 to get financial statements for Tesla
\end{itemize}
\end{frame}

\section{Key Financial Ratios}

\begin{frame}{Categories of Ratios}
\begin{itemize}
\item Profitability
\item Liquidity
\item Solvency
\item Efficiency
\item Valuation
\end{itemize}
\end{frame}

%----------------------------------------------------------
% Liquidity Ratios
%----------------------------------------------------------
\begin{frame}{Liquidity Ratios}
  \begin{itemize}
    \item Current Ratio: 
    \[
      \frac{\text{Current Assets}}{\text{Current Liabilities}}
    \]

    \item Quick (Acid-Test) Ratio: 
    \[
      \frac{\text{Current Assets} - \text{Inventories}}{\text{Current Liabilities}}
    \]

    \item Cash Ratio: 
    \[
      \frac{\text{Cash \& Cash Equivalents}}{\text{Current Liabilities}}
    \]

    \item Operating Cash Flow Ratio: 
    \[
      \frac{\text{Operating Cash Flow}}{\text{Current Liabilities}}
    \]
  \end{itemize}
\end{frame}

%----------------------------------------------------------
% Solvency (Leverage) Ratios
%----------------------------------------------------------
\begin{frame}{Solvency (Leverage) Ratios}
  \begin{itemize}
    \item Debt Ratio: 
    \[
      \frac{\text{Total Liabilities}}{\text{Total Assets}}
    \]

    \item Debt-to-Equity (D/E) Ratio:
    \[
      \frac{\text{Total Liabilities}}{\text{Shareholders' Equity}}
    \]

    \item Interest Coverage Ratio: 
    \[
      \frac{\text{EBIT}}{\text{Interest Expense}}
    \]

    \item Debt Service Coverage Ratio (DSCR): 
    \[
      \frac{\text{Operating Income}}{\text{Total Debt Service}}
    \]
  \end{itemize}
\end{frame}

%----------------------------------------------------------
% Profitability Ratios
%----------------------------------------------------------
\begin{frame}{Profitability Ratios}
  \begin{itemize}
    \item Gross Profit Margin:
    \[
      \frac{\text{Gross Profit}}{\text{Revenue}}
    \]

    \item Operating Profit Margin:
    \[
      \frac{\text{Operating Profit}}{\text{Revenue}}
    \]

    \item Net Profit Margin:
    \[
      \frac{\text{Net Income}}{\text{Revenue}}
    \]

    \item Return on Assets (ROA):
    \[
      \frac{\text{Net Income}}{\text{Total Assets}}
    \]

    \item Return on Equity (ROE):
    \[
      \frac{\text{Net Income}}{\text{Shareholders' Equity}}
    \]
  \end{itemize}
\end{frame}

%----------------------------------------------------------
% Efficiency (Activity) Ratios
%----------------------------------------------------------
\begin{frame}{Efficiency (Activity) Ratios}
  \begin{itemize}
    \item Asset Turnover Ratio:
    \[
      \frac{\text{Net Sales}}{\text{Average Total Assets}}
    \]

    \item Inventory Turnover Ratio:
    \[
      \frac{\text{Cost of Goods Sold}}{\text{Average Inventory}}
    \]

    \item Days Inventory Outstanding (DIO):
    \[
      365 \div \text{Inventory Turnover}
    \]

    \item Accounts Receivable Turnover:
    \[
      \frac{\text{Net Credit Sales}}{\text{Average Accounts Receivable}}
    \]

    \item Days Sales Outstanding (DSO):
    \[
      \left(\frac{\text{Accounts Receivable}}{\text{Total Credit Sales}}\right)
      \times 365
    \]
  \end{itemize}
\end{frame}
\begin{frame}{More Efficiency (Activity) Ratios}
  \begin{itemize}
    \item Days Payables Outstanding (DPO):
    \[
      \left(\frac{\text{Accounts Payable}}{\text{Cost of Goods Sold}}\right)
      \times 365
    \]

    \item Cash Conversion Cycle (CCC):
    \[
      \text{DSO} + \text{DIO} - \text{DPO}
    \]
  \end{itemize}
\end{frame}

%----------------------------------------------------------
% Investment & Valuation Ratios
%----------------------------------------------------------
\begin{frame}{Investment \& Valuation Ratios}
  \begin{itemize}
    \item Price-to-Earnings (P/E) Ratio:
    \[
      \frac{\text{Market Share Price}}{\text{Earnings Per Share}}
    \]

    \item Price-to-Book (P/B) Ratio:
    \[
      \frac{\text{Market Price per Share}}{\text{Book Value per Share}}
    \]

    \item Price-to-Sales (P/S) Ratio:
    \[
      \frac{\text{Market Capitalization}}{\text{Annual Sales}}
    \]

    \item Price-to-Cash Flow (P/CF) Ratio:
    \[
      \frac{\text{Market Price per Share}}{\text{Cash Flow per Share}}
    \]

    \item Enterprise Value to EBITDA (EV/EBITDA):
    \[
      \frac{\text{Enterprise Value}}{\text{EBITDA}}
    \]
  \end{itemize}
\end{frame}

\begin{frame}{More Investment \& Valuation Ratios}
  \begin{itemize}
    \item PEG Ratio:
    \[
      \frac{\text{P/E}}{\text{Annual EPS Growth Rate}}
    \]

    \item Dividend Yield:
    \[
      \frac{\text{Annual Dividends per Share}}{\text{Share Price}}
    \]

    \item Dividend Payout Ratio:
    \[
      \frac{\text{Annual Dividends per Share}}{\text{Earnings per Share}}
    \]
  \end{itemize}
\end{frame}

\section{Julius Workflow}

\begin{frame}{Julius Workflow}
\begin{itemize}
    \item User prompt: input ticker symbol
    \item Julius prompt:
\begin{itemize}
\item Get several years of statements
\item Calculate ratios each year (can skip Investment \& Valuation)
\item Plot ratios over time
\begin{itemize}
\item Maybe one figure for each category?
\item Subplot for each ratio?
\end{itemize}
\item Describe the trends in each category
\item Save as a Word doc?
\end{itemize}
\end{itemize}
\end{frame}

\section{More on EDGAR}

\begin{frame}{Scraping EDGAR}
\begin{itemize}
\item We can retrieve all of the filings of a company by 
\begin{itemize}
\item Find the company's CIK (web search)
\item Add zeros to the beginning of the CIK until it is 10 digits
\item Paste the 10-digit CIK into https://data.sec.gov/api/xbrl/companyfacts/CIK\{cik\_goes\_here\}.json without the braces and visit the URL
\end{itemize}
\item Ask Julius to do this.  We will want the us-gaap data.  Filter to 10Ks or 10Qs or by year and create a dataframe from the data.
\end{itemize}
\end{frame}


\end{document}
