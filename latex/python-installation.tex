\documentclass[10pt]{beamer}
\usepackage{verbatim}
\usepackage{hyperref}
\hypersetup{
    colorlinks=true,
    linkcolor=black,
    filecolor=black,
    urlcolor=blue,
    citecolor=black
}

\usetheme[progressbar=foot]{metropolis}
\usepackage{appendixnumberbeamer}

\usepackage{booktabs}
\usepackage[scale=2]{ccicons}

\usepackage{pgfplots}
\usepgfplotslibrary{dateplot}

\usepackage{xspace}
\usepackage{xcolor}

\DeclareMathOperator{\stdev}{stdev}
\DeclareMathOperator{\var}{var}
\DeclareMathOperator{\cov}{cov}
\DeclareMathOperator{\corr}{corr}
\DeclareMathOperator{\prob}{prob}
\DeclareMathOperator{\n}{n}
\DeclareMathOperator{\N}{N}
\DeclareMathOperator{\Cov}{Cov}

\newcommand{\D}{\mathrm{d}}
\newcommand{\E}{\mathrm{e}}
\newcommand{\mye}{\ensuremath{\mathsf{E}}}
\newcommand{\myreal}{\ensuremath{\mathbb{R}}}

\setbeamertemplate{frame footer}{MGMT 675}


\setbeamertemplate{title page}{
  \begin{centering}
    \begin{beamercolorbox}[sep=8pt,center]{title}
      \usebeamerfont{title}\inserttitle\par%
      \ifx\insertsubtitle\@empty%
      \else%
        \vskip0.25em%
        {\usebeamerfont{subtitle}\usebeamercolor[fg]{subtitle}\insertsubtitle\par}%
      \fi%     
    \end{beamercolorbox}%
    \vfill
    \begin{beamercolorbox}[sep=8pt,center]{date}
      \usebeamerfont{date}\insertdate
    \end{beamercolorbox}
    \vskip0.5em
    {\usebeamercolor[fg]{titlegraphic}\inserttitlegraphic\par}
  \end{centering}
}

\title{Installing Python Locally}
\subtitle{MGMT 675: AI-Assisted Financial Analysis}
\titlegraphic{\includegraphics[height=1cm]{../docs/RiceBusiness-transparent-logo-sm.png}}
\date{}
\begin{document}

\begin{frame}[plain]
\titlepage
\end{frame}

\begin{frame}{Outline}
\begin{itemize}
\item Install python
\item Install packages including jupyterlab
\item Test installation
\item Run Streamlit app
\item Use JupyterLab
\end{itemize}
\end{frame}

\begin{frame}{Install Python}
\begin{itemize}
    \item Download installer from python.org
    \item Caution: When running installer, check "Add Python to PATH" \alert{on very first screen}.
    \item Run installer, accepting other defaults.
\end{itemize}
\end{frame}

\begin{frame}{Install Packages}
\begin{itemize}
    \item Open Command Prompt app (Terminal on Mac)
    \item Pip install libraries.  Can install libraries one at a time or in a batch by typing in Command Prompt and hitting enter.
    \item Example: 
    \begin{itemize}
        \item \texttt{pip install pandas numpy matplotlib yfinance==0.2.54}
        \item \texttt{pip install streamlit==1.44.0 seaborn jupyterlab}
        \item \texttt{pip install scipy statsmodels python-pptx lxml}
        \item \texttt{pip install openpyxl pandas-datareader}
    \end{itemize}
\end{itemize}
\end{frame}

\begin{frame}{Test Installation}
\begin{itemize}
    \item On Windows, in Windows Explorer (file/directory app), select "This PC $>$ Local Disk (C:) $>$ Users $>$ Your username"
    \item Click New - Directory and name it mgmt675 
    \item Download test.py from the course website to the mgmt675 directory
    \item In Command Prompt, execute 
  
    \texttt{cd "C:\textbackslash Users\textbackslash Your username\textbackslash mgmt675" }

    \item In Command Prompt, execute 
   
    \texttt{python test.py}

  On Mac, use \texttt{python3 test.py}
    \item You should see "Ready to Go!"  If you get a "no module named ..." error, use pip install to install the missing module.
    \end{itemize}
\end{frame}

\begin{frame}{Run Streamlit App}
\begin{itemize}
    \item Download "Streamlit zipfile" from course website.  Extract all files to the mgmt675 directory.
    \item Navigate using cd to the mgmt675 directory in Command Prompt if you are not already there.
    \item In Command Prompt, execute "streamlit run app.py."
    \item The app should open in a tab in your default browser.
    \item When you enter a ticker, the PowerPoint deck should be downloaded to the mgmt675 directory.
\end{itemize}
\end{frame}

\begin{frame}{Use JupyterLab}
    \begin{itemize}
    \item Open Command Prompt and use cd to navigate to the mgmt675 directory if you are not already there.
    \item In Command Prompt, execute "jupyter lab"
    \item The JupyterLab app should open in a tab in your default browser.
    \item Test with, for example, \texttt{import numpy as np} and \texttt{np.sqrt(9)} on two separate lines.
    \end{itemize}
\end{frame}

\end{document}