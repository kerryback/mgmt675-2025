\documentclass[10pt]{beamer}

\usepackage{hyperref}
\hypersetup{
    colorlinks=true,
    linkcolor=black,
    filecolor=black,
    urlcolor=blue,
    citecolor=black
}


\usetheme[progressbar=foot]{metropolis}
\usepackage{appendixnumberbeamer}

\usepackage{booktabs}
\usepackage[scale=2]{ccicons}

\usepackage{pgfplots}
\usepgfplotslibrary{dateplot}

\usepackage{xspace}
\usepackage{xcolor}

\DeclareMathOperator{\stdev}{stdev}
\DeclareMathOperator{\var}{var}
\DeclareMathOperator{\cov}{cov}
\DeclareMathOperator{\corr}{corr}
\DeclareMathOperator{\prob}{prob}
\DeclareMathOperator{\n}{n}
\DeclareMathOperator{\N}{N}
\DeclareMathOperator{\Cov}{Cov}

\newcommand{\D}{\mathrm{d}}
\newcommand{\E}{\mathrm{e}}
\newcommand{\mye}{\ensuremath{\mathsf{E}}}
\newcommand{\myreal}{\ensuremath{\mathbb{R}}}

\setbeamertemplate{frame footer}{MGMT 675}


\setbeamertemplate{title page}{
  \begin{centering}
    \begin{beamercolorbox}[sep=8pt,center]{title}
      \usebeamerfont{title}\inserttitle\par%
      \ifx\insertsubtitle\@empty%
      \else%
        \vskip0.25em%
        {\usebeamerfont{subtitle}\usebeamercolor[fg]{subtitle}\insertsubtitle\par}%
      \fi%     
    \end{beamercolorbox}%
    \vfill
    \begin{beamercolorbox}[sep=8pt,center]{date}
      \usebeamerfont{date}\insertdate
    \end{beamercolorbox}
    \vskip0.5em
    {\usebeamercolor[fg]{titlegraphic}\inserttitlegraphic\par}
  \end{centering}
}

\title{Simulation and Forecasting}
\subtitle{MGMT 675: AI-Assisted Financial Analysis}
\titlegraphic{\includegraphics[height=1cm]{../docs/RiceBusiness-transparent-logo-sm.png}}
\date{}
\begin{document}

\begin{frame}[plain]
\titlepage
\end{frame}

\begin{frame}{Outline}
\begin{itemize}
\item Setting up a simulation in Julius/python
\item Random walks
\item Mean reversion
\item Estimating parameters from historical data
\item Forecasting the crude oil price
\item Julius workflow
\end{itemize}
\end{frame}

\section{Simulating in Julius}

\begin{frame}{Setting up a simulation in Julius/python}
\begin{itemize}
\item Model calculates outputs given inputs
\item Ask Julius to create a function that calculates outputs given inputs
\item Ask Julius to run a given number of simulations drawing inputs from given distributions
\item Ask Julius to describe the distributions of the outputs: histogram, mean, median, standard deviation, 90\% confidence interval, etc.
\end{itemize}
\end{frame}

\begin{frame}{Retirement planning}
\begin{itemize}
\item Inputs:
\item \pause Treat array of investment returns as an input
\item \pause To simulate, ask Julius to create the array in each simulation as an array of independent random variables drawn from some distribution.
\end{itemize}
\end{frame}

\section{Random Walks and Mean Reversion}
\begin{frame}{Random walks}
\begin{itemize}
\item A random walk is a variable for which the \alert{changes} or \alert{percent changes} are independent random variables.
\item Example
\begin{itemize}
\item Model stock returns as independent random variables
\item Implies stock price is a random walk
\end{itemize}
\item Interest rates are very close to random walks.
\end{itemize}
\end{frame}

\begin{frame}{Mean reversion}
\begin{itemize}
\item A variable is mean reverting if changes are usually partially reversed by subsequent changes.
\item "What goes up must come down" 
\item Interest rates don't get infinitely large.  Positive changes tend to be somewhat reversed by subsequent negative changes.
\item Commodity prices (crude oil, etc.) tend to be mean reverting.
\end{itemize}
\end{frame}

\begin{frame}{A simple mean-reverting variable}
\begin{itemize}
\item Simple example of mean reverting variable is
$$x_{t+1} = a + b x_t + \epsilon_{t+1}$$
where $\epsilon$'s are iid with mean 0.
\item $E[x_{t+1} \mid x_t] = a + b x_t$.
\item $E[\Delta x_{t+1} \mid x_t] = a + (b-1) x_t$.
\item Assume $b<1$.  Then,
\begin{itemize}
\item $E[\Delta x_{t+1} \mid x_t] > 0 \;\Leftrightarrow\; x_t < a/(1-b)$.
\item $E[\Delta x_{t+1} \mid x_t] < 0 \;\Leftrightarrow\; x_t > a/(1-b)$.
\item $a/(1-b)$ is the long-run mean of $x$.
\end{itemize}
\end{itemize}
\end{frame}

\section{Example}

\begin{frame}{Simulating the Price of Crude Oil}
\begin{itemize}
\item Ask Julius to get monthly crude oil prices from FRED for the longest possible history using pandas data reader.
\item Ask Julius to regress the crude oil price on the lagged price.
\item Ask Julius to compute intercept / (1 - slope coefficient).
\item Ask Julius to plot the crude oil prices and include a horizontal line at intercept / (1 - slope coefficient).
\item Ask Julius to simulate from the model assuming normally distributed residuals and plot the distribution of the crude oil price 24 months from today.
\item Warning: Julius may talk about AR(1) models and change the notation.
\end{itemize}
\end{frame}

\section{Julius Workflow}

\begin{frame}{Retirement planning workflow: User prompt}
Specify in real terms (today's dollars)
\begin{itemize}
\item  current investment account balance, 
\item number of years before retirement
\item planned annual savings amounts before retirement, 
\item expected annual investment return and standard deviation 
\item number of years of withdrawals after retirement
\item planned annual withdrawals after retirement
\item borrowing rate if balance goes negative
\end{itemize}
\end{frame}

\begin{frame}{Retirement planning workflow: Julius prompt}
\begin{itemize}
\item Simulate returns by year 
\item Update the investment account balance each year based on the planned deposit or withdrawal and the simulated return (using the borrowing rate instead if the balance is negative)
\item Repeat 1,000 times
\item Report the probability that the plan is successful (ending account balance is not negative).
\item Compute the median (across simulations) account balance each year and plot by year.
\item Produce a histogram of the ending account balance across simulations 
\end{itemize}
\end{frame}


\end{document}
