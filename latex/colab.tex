\documentclass[10pt]{beamer}
\usepackage{verbatim}
\usepackage{hyperref}
\hypersetup{
    colorlinks=true,
    linkcolor=black,
    filecolor=black,
    urlcolor=blue,
    citecolor=black
}

\usetheme[progressbar=foot]{metropolis}
\usepackage{appendixnumberbeamer}

\usepackage{booktabs}
\usepackage[scale=2]{ccicons}

\usepackage{pgfplots}
\usepgfplotslibrary{dateplot}

\usepackage{xspace}
\usepackage{xcolor}

\DeclareMathOperator{\stdev}{stdev}
\DeclareMathOperator{\var}{var}
\DeclareMathOperator{\cov}{cov}
\DeclareMathOperator{\corr}{corr}
\DeclareMathOperator{\prob}{prob}
\DeclareMathOperator{\n}{n}
\DeclareMathOperator{\N}{N}
\DeclareMathOperator{\Cov}{Cov}

\newcommand{\D}{\mathrm{d}}
\newcommand{\E}{\mathrm{e}}
\newcommand{\mye}{\ensuremath{\mathsf{E}}}
\newcommand{\myreal}{\ensuremath{\mathbb{R}}}

\setbeamertemplate{frame footer}{MGMT 675}


\setbeamertemplate{title page}{
  \begin{centering}
    \begin{beamercolorbox}[sep=8pt,center]{title}
      \usebeamerfont{title}\inserttitle\par%
      \ifx\insertsubtitle\@empty%
      \else%
        \vskip0.25em%
        {\usebeamerfont{subtitle}\usebeamercolor[fg]{subtitle}\insertsubtitle\par}%
      \fi%     
    \end{beamercolorbox}%
    \vfill
    \begin{beamercolorbox}[sep=8pt,center]{date}
      \usebeamerfont{date}\insertdate
    \end{beamercolorbox}
    \vskip0.5em
    {\usebeamercolor[fg]{titlegraphic}\inserttitlegraphic\par}
  \end{centering}
}

\title{Google Colab}
\subtitle{MGMT 675: AI-Assisted Financial Analysis}
\titlegraphic{\includegraphics[height=1cm]{../docs/RiceBusiness-transparent-logo-sm.png}}
\date{}
\begin{document}

\begin{frame}[plain]
\titlepage
\end{frame}

\begin{frame}
    \frametitle{DownloadTesla Form 4 notebook}
    \begin{itemize}
        \item Install Google Drive \href{https://support.google.com/a/users/answer/13022292?hl=en}{here} if you don't have it already.      
        \item Download the Tesla Form 4 notebook from the course websiteto the MyDrive folder in Google Drive.
    \end{itemize}
\end{frame}

\begin{frame}{Google Colab}
    \begin{itemize}
        \item Go to \href{https://colab.research.google.com}{https://colab.research.google.com}.
        \item You should see a Welcome screen with an Open Notebook popup.
        \item Click on New Notebook.
  \end{itemize}
\end{frame}

\begin{frame}{Notebooks}
    \begin{itemize}
        \item A Jupyter notebook consists of code cells and text (markdown) cells.
        \item Code cells can be executed.
        \item Text cells can be edited as markdown.
        \item You get free access to Gemini to help you write code.
        \item You can add new cells by clicking the + button in the toolbar or the popup below a cell.
        \item The play button will execute the code in the selected cell.
        \item Type 
        
        \texttt{import numpy as np}

        \texttt{np.sqrt(9)}

        and run the cell to see the result.
        \item Watch the video linked on the course website or one of many youtube videos for a quick introduction.
        
    \end{itemize}
\end{frame}

\begin{frame}{Run Tesla Form 4 notebook}
    \begin{itemize}
        \item In Colab, click on File - Open notebook.
        \item In the left sidebar, click on Google Drive.
        \item Click on the Tesla Form 4 notebook to open it.
        \item  Click Runtime - Run all to run the entire notebook.
    \end{itemize}
\end{frame}

\end{document}