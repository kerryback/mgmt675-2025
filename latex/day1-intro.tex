\documentclass[10pt]{beamer}

\usepackage{hyperref}
\hypersetup{
    colorlinks=true,
    linkcolor=black,
    filecolor=black,
    urlcolor=blue,
    citecolor=black
}


\usetheme[progressbar=foot]{metropolis}
\usepackage{appendixnumberbeamer}

\usepackage{booktabs}
\usepackage[scale=2]{ccicons}

\usepackage{pgfplots}
\usepgfplotslibrary{dateplot}

\usepackage{xspace}
\usepackage{xcolor}

\DeclareMathOperator{\stdev}{stdev}
\DeclareMathOperator{\var}{var}
\DeclareMathOperator{\cov}{cov}
\DeclareMathOperator{\corr}{corr}
\DeclareMathOperator{\prob}{prob}
\DeclareMathOperator{\n}{n}
\DeclareMathOperator{\N}{N}
\DeclareMathOperator{\Cov}{Cov}

\newcommand{\D}{\mathrm{d}}
\newcommand{\E}{\mathrm{e}}
\newcommand{\mye}{\ensuremath{\mathsf{E}}}
\newcommand{\myreal}{\ensuremath{\mathbb{R}}}

\newcommand{\eqdef}{\;\buildrel \text{d{}ef}\over = \;}

\setbeamertemplate{frame footer}{MGMT 675, Rice University}


\setbeamertemplate{title page}{
  \begin{centering}
    \begin{beamercolorbox}[sep=8pt,center]{title}
      \usebeamerfont{title}\inserttitle\par%
      \ifx\insertsubtitle\@empty%
      \else%
        \vskip0.25em%
        {\usebeamerfont{subtitle}\usebeamercolor[fg]{subtitle}\insertsubtitle\par}%
      \fi%     
    \end{beamercolorbox}%
    \vfill
    \begin{beamercolorbox}[sep=8pt,center]{date}
      \usebeamerfont{date}\insertdate
    \end{beamercolorbox}
    \vskip0.5em
    {\usebeamercolor[fg]{titlegraphic}\inserttitlegraphic\par}
  \end{centering}
}

\title{Course Overview}
\subtitle{MGMT 675: AI-Assisted Financial Analysis}
\titlegraphic{\includegraphics[height=1cm]{../docs/RiceBusiness-transparent-logo-sm.png}}
\date{}
\begin{document}

\begin{frame}[plain]
\titlepage
\end{frame}

\begin{frame}{Meet your Prof}
\begin{itemize}
\item At Rice since 2009, in Jones and in Econ Dept.
  \begin{itemize}
  \item teach core finance, quantitative investments, investments theory, and python for business research
  \item to PhD and Masters in Data Science students
  \end{itemize}
\item Previously at Northwestern, Indiana, Washington Univ. in St. Louis, and Texas A\&M. Associate Dean at Wash U.
\item Former and current editor and associate editor of several journals. Two textbooks (derivatives and PhD asset pricing theory). 
\item Course materials and info at \href{https://kerryback.com}{kerryback.com}
\end{itemize}
\end{frame}

\section{Course Overview}

\begin{frame}{Motivation for this course}
\begin{itemize}
\item \href{https://www.cio.com/article/2089550/ai-poised-to-replace-entry-level-positions-at-large-financial-institutions.html}{How will AI affect finance jobs?}

\item Basic template for finance chatbot
  \begin{itemize}
  \item Python code for financial analysis
  \item App that encapsulates python code
  \item App makes API calls to LLMs to process user's (natural language) input and pass structured version to python code
  \end{itemize}

\item Random web cites: 40+\% of all new code written by AI, 80+\% of developers use AI assistance
\end{itemize}
\end{frame}

\begin{frame}{Learning objectives}
\begin{itemize}
\item Learn to use AI to write python code to perform financial analyses
\item Get additional practice in financial analyses
\item Obtain a basic understanding of
  \begin{itemize}
  \item Python
  \item Making API calls to LLMs
  \item Building simple apps
  \end{itemize}
\end{itemize}
\end{frame}


\begin{frame}{Platforms}
\begin{itemize}
\item Julius.ai to write and execute python code
\item Google Colab as alternative python environment (free)
\item VS Code (free) and Cursor as alternative local environments
\item Streamlit Cloud for hosting apps (free)
\item OpenAI for API calls
\item HuggingFace as alternative for API calls (free)
\end{itemize}
\end{frame}

\begin{frame}{Platforms}
\begin{itemize}
\item \alert{Julius.ai to write and execute python code}
\item Google Colab as alternative python environment (free)
\item VS Code (free) and Cursor as alternative local environments
\item Streamlit Cloud for hosting apps (free)
\item OpenAI for API calls
\item HuggingFace as alternative for API calls (free)
\end{itemize}
\end{frame}
\begin{frame}{Assignments}
\begin{itemize}
\item 6 group assignments, due by midnight on Thursdays beginning this week
\item Form groups on canvas of no more than six students
\item We'll break out into groups on most days
\item Each assignment is to provide a link to a Julius workflow that will accomplish a specific financial analysis

\end{itemize}
\end{frame}

\section{Julius}
\begin{frame}{Get a Julius account}
\begin{itemize}
\item Julius.ai provides a 50\% academic discount. Sign up for a free account, then send an email using your Rice email account to team@julius.ai and ask for the academic discount. They will respond with a promo code to use.
\item The Lite account (\$8 per month after discount) allows 250 messages per month and may be ok.
\item Standard account (\$18 per month after discount) allows unlimited messages.
\end{itemize}
\end{frame}

\begin{frame}{Warm-Up}
  Click "+ New" to start a chat (thread)with Julius.  Ask it to do the following (you could start a new chat for each topic):
\begin{itemize}
\item Ask Julius to use yfinance 0.2.54 to get AAPL's closing price from Yahoo Finance, plot it, and save as a jpeg.
\item Ask Julius to use pandas datareader to get the 10-year Treasury yield from FRED, plot it, and save as a jpeg.
\item Ask Julius to plot the payoff diagram of a call option with a strike of 100.
\item Ask Julius what data it needs to compute the Black-Scholes value of a call option. Supply the data and ask for the value.
\end{itemize}
\end{frame}

\begin{frame}{Julius Workflows}
  \begin{itemize}
  \item Scroll down the "Workspace" menu to find and select "Workflows." 
  \item Move to the "Sandbox" tab and select "New Sandbox."
  \item The "Prompt" menu on the right includes a "User Prompt" option and a "Prompt" (tell Julius to run an analysis) option.
  \item Each workflow should start with a user prompt, asking the user to supply data and maybe asking the user what output the user wants.
  \item Specify in a Julius prompt the steps Julius should take with the information provided by the user.  
  \item Follow up with another user prompt if needed, etc.
  \end{itemize}
\end{frame}

\begin{frame}{Example Workflow}
  Create a workflow including the following:
  \begin{itemize}
  \item User prompt: Ask the user for a ticker and a start date.
  \item Julius prompt: Use yfinance to get the closing price of the stock from Yahoo Finance beginning at the start date.  Plot the price.  Output the plot to a jpeg file and show it to the user.
  \end{itemize}
\end{frame}

\begin{frame}{Effective Prompting}
\begin{itemize}
\item We are writing prompts to Julius that will supplement user input.
\item Be explicit in what you want.  Break down into steps if possible.
\item Provide examples to Julius (code that has worked before or spreadsheets).\item Ask Julius to explain each step as it accomplishes it.
\item Ask Julius to explain what it has done as it accomplishes each step.
\end{itemize}
\end{frame}

\section{Excel, Python, and AI}

\begin{frame}{Excel and Python}
\begin{itemize}
  \item Python output $\rightarrow$ Excel data is easy
\item Data in Excel $\rightarrow$ python is easy
\item Formulas in Excel $\rightarrow$ python is possible via AI (ask Julius to read the formulas not the data).
\end{itemize}
\end{frame}

\begin{frame}{Building and saving models in python with AI}
\begin{itemize}
\item Building:
\begin{itemize}
\item Read and mimic Excel model, or
\item Ask AI to build a model from scratch
\item Both are often iterative processes.  Need to correct AI mistakes.
\end{itemize}
\item Saving:
\begin{itemize}
\item Add working code to a Julius workflow: Copy working code and paste it into a "Prompt" cell as an example when creating a workflow.  
\item Save working code elsewhere:
\begin{itemize}
\item Copy and paste into a Jupyter notebook on your hard drive or Google Drive (for Colab).
\item Create a chatbot app that runs the code.
\end{itemize}
\end{itemize}
\end{itemize}
\end{frame}

\end{document}