\documentclass[10pt]{beamer}

\usepackage{hyperref}
\hypersetup{
    colorlinks=true,
    linkcolor=black,
    filecolor=black,
    urlcolor=blue,
    citecolor=black
}

\usetheme[progressbar=foot]{metropolis}
\usepackage{appendixnumberbeamer}

\usepackage{booktabs}
\usepackage[scale=2]{ccicons}

\usepackage{pgfplots}
\usepgfplotslibrary{dateplot}

\usepackage{xspace}
\usepackage{xcolor}

\DeclareMathOperator{\stdev}{stdev}
\DeclareMathOperator{\var}{var}
\DeclareMathOperator{\cov}{cov}
\DeclareMathOperator{\corr}{corr}
\DeclareMathOperator{\prob}{prob}
\DeclareMathOperator{\n}{n}
\DeclareMathOperator{\N}{N}
\DeclareMathOperator{\Cov}{Cov}

\newcommand{\D}{\mathrm{d}}
\newcommand{\E}{\mathrm{e}}
\newcommand{\mye}{\ensuremath{\mathsf{E}}}
\newcommand{\myreal}{\ensuremath{\mathbb{R}}}

\setbeamertemplate{frame footer}{MGMT 675}


\setbeamertemplate{title page}{
  \begin{centering}
    \begin{beamercolorbox}[sep=8pt,center]{title}
      \usebeamerfont{title}\inserttitle\par%
      \ifx\insertsubtitle\@empty%
      \else%
        \vskip0.25em%
        {\usebeamerfont{subtitle}\usebeamercolor[fg]{subtitle}\insertsubtitle\par}%
      \fi%     
    \end{beamercolorbox}%
    \vfill
    \begin{beamercolorbox}[sep=8pt,center]{date}
      \usebeamerfont{date}\insertdate
    \end{beamercolorbox}
    \vskip0.5em
    {\usebeamercolor[fg]{titlegraphic}\inserttitlegraphic\par}
  \end{centering}
}

\title{Mean-Variance Optimization}
\subtitle{MGMT 675: AI-Assisted Financial Analysis}
\titlegraphic{\includegraphics[height=1cm]{../docs/RiceBusiness-transparent-logo-sm.png}}
\date{}
\begin{document}

\begin{frame}[plain]
\titlepage
\end{frame}
\begin{frame}{Outline}
\begin{itemize}
\item Finding the capital allocation line by solving equations
\item Finding the hyperbola of efficient portolios of only risky assets
\item Excel example 
\item Julius workflow
\end{itemize}
\end{frame}


\section{Solving Equations for Mean-Variance Optimization}

\begin{frame}{Mean-variance optimization}
\begin{itemize}
\item The following are essentially equivalent procedures:
\begin{enumerate}
\item Maximize Sharpe ratio = risk premium / risk
\item Maximize expected return subject to not exceeding a risk limit 
\item Minimize risk subject to achieving a target expected return
\item Maximize expected return - penalty parameter times variance
\end{enumerate}
\item We can run a solver or solve equations to get the solutions.
\end{itemize}
\end{frame}

\begin{frame}{Calculus review}
\begin{itemize}
\item Suppose $y = x^2 - 2x + 5$ and you want to find the value of $x$ that minimizes $y$.
\item The minimum occurs at the bottom of the curve, where the slope is zero.
  \begin{itemize}
  \item The slope is the derivative, and the derivative is $d y/d x = 2x - 2$.
  \item So, the slope is zero when $2x-2=0$, which implies $x=1$.
  \end{itemize}
\item Bottom line: can solve an equation to find the minimum.
\end{itemize}
\end{frame}

\begin{frame}{Equations that solve mean-variance problems}
\begin{itemize}
\item $w_i=$ weight, $\bar r_i=$ expected return, $r_f=$ risk-free rate, $\sigma_i^2=$ variance, $\sigma_{ik} =$ covariance.
\item 3 asset example: 3 equations in 3 unknowns $w_1, w_2, w_3$
\end{itemize}

$$\sigma_1^2 w_1 + \sigma_{12} w_2 + \sigma_{13}w_3 = \bar r_1 - r_f$$
$$\sigma_{21} w_1 + \sigma_2^2 w_2 + \sigma_{23}w_3 = \bar r_2 - r_f$$
$$\sigma_{31} w_1 + \sigma_{32} w_2 + \sigma_3^2w_3 = \bar r_3 - r_f$$
\end{frame}

\begin{frame}{Matrix multiplication}
    This is not essential, but it will help to understand code.
    \begin{itemize}
    \item Three equations are represented as 
    $$\Sigma w = \bar r - r_f$$
    where $\Sigma = 3 \times 3$ array of covariances and variances, $w=(w_1, w_2, w_3)$ and $\bar r - r_f = (\bar r_1-r_f, \bar r_2 - r_f, \bar r_3 - r_f)$
    \item Solution is represented as 
    $$w = \Sigma^{-1}(\bar r - r_f)$$
    \end{itemize}
    \end{frame}

\begin{frame}{Tangency and other portfolios}
\begin{itemize}
\item Given solution $(w_1, w_2, w_3)$ of the equations, divide by sum of $w_i$ to get tangency portfolio
\item Given tangency portfolio,
\begin{itemize}
    \item Given risk limit (standard deviation), optimal portfolio satisfying the risk limit is
    $$\frac{\text{risk limit}}{\text{std dev of tangency portfolio}} \times \text{tangency portfolio}.$$
    \item Given target expected return, optimal portfolio achieving the target is
    $$\frac{\text{target risk premium}}{\text{risk prem of tangency portfolio}} \times \text{tangency portfolio}$$
    \end{itemize}
\end{itemize}
\end{frame}


\section{Excel Example}

\begin{frame}{Excel Spreadsheet Example}

Excel Example linked on Schedule page

based on Applied Finance Topic5.1\_SharpeRatioExamples.xlsx
\end{frame}

\section{Risky Assets Only}

\begin{frame}{Finding the Markowitz Bullet (Hyperbola)}
\begin{itemize}
\item We can also solve equations to find
\begin{itemize}
\item The global minimum variance portfolio
\item Another portfolio of risky assets on the hyperbola
\end{itemize}
\item And all portfolios on the hyperbola are combinations of those two portfolios.
\item However, Julius will probably run a solver if you ask it to find a portfolio of risky assets with weights summing to 1 that minimizes variance subject to a target expected return.
\end{itemize}
\end{frame}

\section{Julius Workflow}

\begin{frame}{Julius Workflow}
  \begin{itemize}
  \item User prompt:
  \begin{itemize}
  \item Asset names, expected returns, and covariances
  \item Risk-free rate
  \end{itemize}
  \item Julius prompt:
  \begin{itemize}
  \item Compute the tangency portfolio and output to the user
  \item Compute the expected return and standard deviation of the tangency portfolio and output to the user
  \end{itemize}
  \item Optional additional prompt for Julius:
  \begin{itemize}
  \item Compute the minimum risk portfolio with weights that sum to 1 (fully invested in risky assets) for various expected return targets.
  \item  Plot the means and standard deviations of the minimum risk portfolios. 
  \item Show the tangency portfolio and capital allocation line on the plot.  
  \item Display the figure and return a jpeg of it.
  \end{itemize}
\end{itemize}
  \end{frame}
  
  \begin{frame}{Checking Julius' Work}
    \begin{itemize}
    \item Run the workflow with data from Topic5.1\_SharpeRatioExamples.xlsx from Applied Finance (linked on our Schedule page for Day 1).
    \item Check that the tangency portfolio is correct.
    \item If it is correct, copy the code generated by Julius and paste it into the Julius prompt cell in the workflow.  Recommend that Julius use the code.
    \end{itemize}
  \end{frame}
\end{document}
