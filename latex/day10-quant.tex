\documentclass[10pt]{beamer}

\usepackage{hyperref}
\hypersetup{
    colorlinks=true,
    linkcolor=black,
    filecolor=black,
    urlcolor=blue,
    citecolor=black
}


% Add these packages in the preamble
\usepackage{listings}
\usepackage{xcolor}

% Add this styling configuration
\lstdefinestyle{mystyle}{
    backgroundcolor=\color{gray!10},
    basicstyle=\ttfamily\small,
    breakatwhitespace=false,
    breaklines=true,
    captionpos=b,
    commentstyle=\color{green!60!black},
    keywordstyle=\color{blue},
    stringstyle=\color{red},
    showspaces=false,
    showstringspaces=false,
    showtabs=false,
    tabsize=2,
    frame=single
}
\lstset{style=mystyle}




\usepackage{verbatim}
\usetheme[progressbar=foot]{metropolis}
\usepackage{appendixnumberbeamer}

\usepackage{booktabs}
\usepackage[scale=2]{ccicons}

\usepackage{pgfplots}
\usepgfplotslibrary{dateplot}

\usepackage{xspace}
\usepackage{xcolor}

\usepackage{pifont}
\newcommand{\cmark}{\textcolor{green!70!black}{\ding{51}}} % check mark
\newcommand{\xmark}{\textcolor{red}{\ding{55}}}             % cross mark

\DeclareMathOperator{\stdev}{stdev}
\DeclareMathOperator{\var}{var}
\DeclareMathOperator{\cov}{cov}
\DeclareMathOperator{\corr}{corr}
\DeclareMathOperator{\prob}{prob}
\DeclareMathOperator{\n}{n}
\DeclareMathOperator{\N}{N}
\DeclareMathOperator{\Cov}{Cov}

\newcommand{\D}{\mathrm{d}}
\newcommand{\E}{\mathrm{e}}
\newcommand{\mye}{\ensuremath{\mathsf{E}}}
\newcommand{\myreal}{\ensuremath{\mathbb{R}}}

\setbeamertemplate{frame footer}{MGMT 675}


\setbeamertemplate{title page}{
  \begin{centering}
    \begin{beamercolorbox}[sep=8pt,center]{title}
      \usebeamerfont{title}\inserttitle\par%
      \ifx\insertsubtitle\@empty%
      \else%
        \vskip0.25em%
        {\usebeamerfont{subtitle}\usebeamercolor[fg]{subtitle}\insertsubtitle\par}%
      \fi%     
    \end{beamercolorbox}%
    \vfill
    \begin{beamercolorbox}[sep=8pt,center]{date}
      \usebeamerfont{date}\insertdate
    \end{beamercolorbox}
    \vskip0.5em
    {\usebeamercolor[fg]{titlegraphic}\inserttitlegraphic\par}
  \end{centering}
}


\title{Quantitative Equity Investing}
\subtitle{MGMT 675: AI-Assisted Financial Analysis}
\titlegraphic{\includegraphics[height=1cm]{../docs/RiceBusiness-transparent-logo-sm.png}}
\date{}
\begin{document}

\begin{frame}[plain]
\titlepage
\end{frame}

\begin{frame}{Outline}
    \begin{itemize}
    \item Motivation: Can we profitably trade on quantitative signals?
    \item Example dataset
    \item Returns of portfolios formed by sorting on characteristics
    \item Regressing returns on characteristics at each date
    \item Training a model on past data and sorting on its predictions
    \end{itemize}
\end{frame}

\section{Example Data}[fragile]
\begin{frame}
\begin{itemize}
    \item Weekly data on $\approx$ 1,000 stocks from 2021 to present
    \item Roughly top half of Russell 2000
    \begin{itemize}
    \item Sorted on marketcap on Jan 1, 2021.  Chose stocks from 1,001 through 2,000.  
    \item Followed them to present.  Down to around 800 now due to mergers, etc.
    \end{itemize}
    \item All items are as of the end-of-week market close except \verb!ret!
    \item \verb!lag1! is the return over the week ending on the date shown
    \item \verb!lag4! is the return over the prior 4 weeks including the week ending on the date shown, etc.
    \item \verb!rsi! is the Relative Strength Index 
\item Download stocks.csv from the Schedule page and upload to Julius.  Ask Julius to describe the data.
\end{itemize}
\end{frame}
\end{document}

