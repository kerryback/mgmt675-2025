\documentclass[10pt]{beamer}

\usepackage{hyperref}
\hypersetup{
    colorlinks=true,
    linkcolor=black,
    filecolor=black,
    urlcolor=blue,
    citecolor=black
}


% Add these packages in the preamble
\usepackage{listings}
\usepackage{xcolor}

% Add this styling configuration
\lstdefinestyle{mystyle}{
    backgroundcolor=\color{gray!10},
    basicstyle=\ttfamily\small,
    breakatwhitespace=false,
    breaklines=true,
    captionpos=b,
    commentstyle=\color{green!60!black},
    keywordstyle=\color{blue},
    stringstyle=\color{red},
    showspaces=false,
    showstringspaces=false,
    showtabs=false,
    tabsize=2,
    frame=single
}
\lstset{style=mystyle}




\usepackage{verbatim}
\usetheme[progressbar=foot]{metropolis}
\usepackage{appendixnumberbeamer}

\usepackage{booktabs}
\usepackage[scale=2]{ccicons}

\usepackage{pgfplots}
\usepgfplotslibrary{dateplot}

\usepackage{xspace}
\usepackage{xcolor}

\usepackage{pifont}
\newcommand{\cmark}{\textcolor{green!70!black}{\ding{51}}} % check mark
\newcommand{\xmark}{\textcolor{red}{\ding{55}}}             % cross mark

\DeclareMathOperator{\stdev}{stdev}
\DeclareMathOperator{\var}{var}
\DeclareMathOperator{\cov}{cov}
\DeclareMathOperator{\corr}{corr}
\DeclareMathOperator{\prob}{prob}
\DeclareMathOperator{\n}{n}
\DeclareMathOperator{\N}{N}
\DeclareMathOperator{\Cov}{Cov}

\newcommand{\D}{\mathrm{d}}
\newcommand{\E}{\mathrm{e}}
\newcommand{\mye}{\ensuremath{\mathsf{E}}}
\newcommand{\myreal}{\ensuremath{\mathbb{R}}}

\setbeamertemplate{frame footer}{MGMT 675}


\setbeamertemplate{title page}{
  \begin{centering}
    \begin{beamercolorbox}[sep=8pt,center]{title}
      \usebeamerfont{title}\inserttitle\par%
      \ifx\insertsubtitle\@empty%
      \else%
        \vskip0.25em%
        {\usebeamerfont{subtitle}\usebeamercolor[fg]{subtitle}\insertsubtitle\par}%
      \fi%     
    \end{beamercolorbox}%
    \vfill
    \begin{beamercolorbox}[sep=8pt,center]{date}
      \usebeamerfont{date}\insertdate
    \end{beamercolorbox}
    \vskip0.5em
    {\usebeamercolor[fg]{titlegraphic}\inserttitlegraphic\par}
  \end{centering}
}


\title{Walmart Valuation}
\subtitle{MGMT 675: AI-Assisted Financial Analysis}
\titlegraphic{\includegraphics[height=1cm]{../docs/RiceBusiness-transparent-logo-sm.png}}
\date{}
\begin{document}

\begin{frame}[plain]
\titlepage
\end{frame}

\begin{frame}{Outline}
  \begin{itemize}
    \item Review of Two/Three Stage Growth Model
    \item When a Firm Only Earns its Cost of Capital
    \item Review of Enterprise Valuation
    \item Walmart Valuation
  \end{itemize}
\end{frame}

\section{Two/Three Stage Growth Model}

\begin{frame}{Last Stage}
    \begin{itemize}
    \item Assume everything grows at the same rate $g$ from year $T$ onwards.
    \item Then the value of the firm at $T$ is
    \[
        V_T = \frac{D_{T+1}}{k - g}
    \]
    where $D_{T+1}$ is the dividend in year $T+1$ and $k$ is the discount rate.
    \item Define the payout rate $\pi = D_{T+1}/E_{T+1}$, where $E_{T+1}$ is earnings in year $T+1$.
    \item Then $D_{T+1} = \pi E_{T+1}$.
    \item Substituting, we get
    \[
        V_T = \frac{\pi E_{T+1}}{k - g}
    \]
    \end{itemize}
\end{frame}

    \begin{frame}{Value to Book Ratio}
        \begin{itemize}
    \item Define ROE as $\text{ROE} = E_{T+1}/B_T$ where $B_T$ is the book value of the firm at time $T$. Then $E_{T+1} = B_T \times \text{ROE}$.
    \item Substituting, we get
    \[
        V_T = \frac{\pi B_T \times \text{ROE}}{k - g}
    \]
    \item So, the value to book ratio is 
    $$\frac{V_T}{B_T} = \frac{\pi \times \text{ROE}}{k - g}$$
        \end{itemize}
      \end{frame}

    \begin{frame}{ROE and Growth}
      \begin{itemize}
    \item You probably know the formula $g = (1-\pi) \times \text{ROE}$.
    \item This can be seen easily by looking at the growth in book equity.
      \begin{itemize}
    \item The change in book equity is 
    $$B_{T+1} - B_T = E_{T+1} - D_{T+1} = (1-\pi)E_{T+1} = (1-\pi)B_T \times \text{ROE}$$
    \item Therefore, the growth rate is 
    $$g = \frac{B_{T+1}-B_T}{B_T} = (1-\pi) \times \text{ROE}$$
      \end{itemize}
    \end{itemize}
    \end{frame}

    \begin{frame}{Summary}
     Substituting for $g$, we get
     \begin{align*}
      V_T &= \frac{\pi E_{T+1}}{k - (1-\pi)\times \text{ROE}}\\
      \frac{V_T}{B_T} &= \frac{\pi \times \text{ROE}}{k - (1-\pi) \times \text{ROE}}
     \end{align*}
    \end{frame}

    \section{When a Firm Only Earns its Cost of Capital}

    \begin{frame}{When a Firm Only Earns its Cost of Capital}
    \begin{itemize}
    \item Suppose the firm only earns its cost of capital in the last stage, meaning $\text{ROE} = k$.
    \item Then, 
   \begin{align*}
   \frac{V_T}{B_T} &= \frac{\pi  k}{k - (1-\pi)  k} = \frac{\pi  k}{\pi  k} = 1\\
   V_T &= \frac{\pi E_{T+1}}{k - (1-\pi)  k} = \frac{\pi  E_{T+1}}{\pi  k}= \frac{E_{T+1}}{k}
  \end{align*}
  \item Payout rate and growth rate do not matter if a firm is earning exactly its cost of capital (NPV of all growth is zero).
   \item Also, we have $E_{T+1} = kB_T \Rightarrow B_T = E_{T+1}/k$.
\end{itemize}
\end{frame}

\begin{frame}{Caution about First (or First and Second) Stage}
    \begin{itemize}
    \item Case Exhibit 4 makes assumptions about earnings growth and payout rates.
    \item Earnings + payout rate $\Rightarrow$ book equity growth
   $$ \Delta \text{Book Equity} = \text{Earnings} - \text{Payouts} $$
    \item Book equity + earnings $\Rightarrow$ ROE.  Should check if implied ROE assumptions are reasonable.
    \item Also, assumptions are likely to be inconsistent.
    \begin{itemize}
    \item We have $E_{T+1} = \text{ROE} \times B_T = k B_T$.
    \item So book equity at $T$ must be $E_{T+1}/k$.  This is unlikely to be true, given starting book equity and implied first-stage book equity growth. \href{https://mgmt675-2025.kerryback.comassets/Walmart_Edited.xlsx}{Exhibit 4}.
    \end{itemize}
    \item I prefer making assumptions about ROE and payout rates instead of growth and payout rates.  Actually, better to make assumptions about the determinants of ROE.  Actually, better to do enterprise valuation.
  \end{itemize}
\end{frame}

\section{Enterprise Valuation}

\begin{frame}{Enterprise Valuation: Definitions}
    \begin{itemize}
        \item \alert{Invested Capital}  $=$ Operating Assets minus Operating Liabilities (more later)
        \item \alert{Net Operating Profit After Taxes} Don't deduct interest in income statement.  Net income replaced by NOPAT (sometimes called NOPLAT or EBIAT)
        \item \alert{Return on Invested Capital} $\text{ROIC}_{t} = \text{NOPAT}_{t}/\text{Invested Capital}_{t-1}$
        \item \alert{Plowback rate} $=\Delta \text{Invested Capital}_{t}/\text{\text{NOPAT}}_{t}$
        \item \alert{Payout rate}  $\pi = 1 - \text{plowback rate}$
        \item \alert{Free cash flow} $=$ NOPAT $-\Delta$ Invested Capital $= \pi \times \text{NOPAT}$
    \end{itemize}
  \end{frame}

  \begin{frame}{Enterprise Value and Equity Value}
    \begin{itemize}
        \item Enterprise Value = present value of free cash flow discounted at WACC
        \item Equity Value = Enterprise Value - Debt - Minority Interests + Value of Equity in Nonconsolidated Subsidiaries
    \end{itemize}
    \end{frame}

    \begin{frame}{Enterprise Valuation: Last Stage}
        \begin{itemize}
         \item In perpetual constant growth, the growth rate is 
         $$g = (1-\pi) \times \text{ROIC}$$
         \item Enterprise value at $T$ is 
         $$\frac{\text{FCF}_{T+1}}{\text{WACC} - g}$$
        \item Earning cost of capital means ROIC $=$ WACC. Then the enterprise value at $T$ is 
        $$\text{Invested Capital}_T = \frac{\text{NOPAT}_{T+1}}{\text{WACC}}$$
        \end{itemize}
\end{frame}

\section{Walmart Valuation}

\begin{frame}{Walmart Valuation}
    \begin{itemize}
    \item \href{https://mgmt675-2025.kerryback.comassets/Walmart.xlsx}{Case data}
    \item \href{https://julius.ai/s/1eb08d2f-4f6e-4d5a-8787-5228da1d22bd}{Julius thread}
    \item \href{https://mgmt675-2025.kerryback.comassets/Walmart_NOPAT_IC.xlsx}{NOPAT and Invested Capital calculation in Excel}
    \item \href{https://mgmt675-2025.kerryback.comassets/Walmart_Forecast_Model.xlsx}{Walmart Forecast Model from Julius}
    \end{itemize}
\end{frame}
\end{document}